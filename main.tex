\documentclass[12pt,twoside]{book}

\usepackage[utf8]{inputenc}
\usepackage{amsmath,amssymb,amsthm,mathrsfs,amsfonts,xfrac,pifont}

\usepackage{array,tabularx}
\usepackage[retainorgcmds]{IEEEtrantools}
\usepackage{emerald,xcolor,tikz,tikz-cd}
\definecolor{lightergray}{rgb}{0.9,0.9,0.9}

\usepackage{enumitem}

\usepackage[
  top=1.25in,
  bottom=1.25in,
  left=1.25in,
  right=1.25in,
  bindingoffset=0in,
  heightrounded,
]{geometry}

\usepackage{fancyhdr}
\pagestyle{fancy}
\fancyhead{}
\fancyhead[LO]{\nouppercase{{\slshape\leftmark}}}
\fancyhead[RO]{\thepage}
\fancyhead[RE]{\nouppercase{{\slshape\rightmark}}}
\fancyhead[LE]{\thepage}
\renewcommand{\headrulewidth}{0.5pt}
\setlength{\headheight}{15pt}
\fancyfoot{}
%\cfoot{\nouppercase{\textbf{\textsc{\textsf{\thepage}}}}}

\usepackage{titlesec}
%\titleformat{hcommandi}[hshapei]{hformati}{hlabeli}{hsepi}{hbefore-codei}[hafter-codei]
%\renewcommand\thepart{\arabic{part}}
\titleformat{\part}[display]{\normalfont}{\hrule height 0.5pt\vfill\Large\itshape\partname\ \thepart}{25pt}{\bfseries\Huge}[\vfill]
\titlespacing*{\part}{0pt}{-35pt}{0pt}%[270pt]
\titleformat{\chapter}[display]
% {\normalfont}{\vspace{-2.875cm}\titlerule[0.5pt] \flushright \vspace{-13pt} \rule{100pt}{7pt}\\ \Large\itshape\chaptertitlename\ \thechapter}{50pt}{\bfseries\Huge}
{\normalfont}{\vspace{-3cm} \leavevmode\leaders\hrule height 0.5pt\hfill \raisebox{-6.5pt}{\rule{100pt}{7pt}} \flushright \Large\itshape\chaptertitlename\ \thechapter}{50pt}{\bfseries\Huge}
\titlespacing*{\chapter} {0pt}{35pt}{40pt}


% \usepackage{tocloft}
% \renewcommand{\cftdot}{ }
% \renewcommand{\cftpartfont}{\bfseries Part }
% \renewcommand{\cftpartaftersnum}{.}
% \renewcommand{\cftchapfont}{Chapter }
% \renewcommand{\cftchapaftersnum}{.}
% \renewcommand{\cftsecfont}{\S }
% \renewcommand{\cftsecaftersnum}{.}

% \usepackage{erewhon,cabin}
% \usepackage{titletoc}

%     \titlecontents{thepart}
%     [5.5em] %5.3
%     {\bigskip}
%     {\bfseries\contentslabel[Part~\thecontentslabel.]{5.5em}}%\thecontentslabel
%     {\hspace*{-5.5em}}% unnumbered chapters
%     {\hfill\contentspage}[\smallskip]%
    
%     \titlecontents{chapter}
%     [5.5em] %5.3
%     {\medskip}
%     {\contentslabel[\chaptername~\thecontentslabel.]{5.5em}}%\thecontentslabel
%     {\hspace*{-5.5em}}% unnumbered chapters
%     {\hfill\contentspage}[\smallskip]%
% %
%     \titlecontents{section}
%     [5.5em] % i
%     {\smallskip}
%     {\S\thecontentslabel. }%\thecontentslabel
%     {}
%     {\hfill\contentspage}%]


\usepackage[colorlinks=true,linkcolor=blue]{hyperref}

%\theoremstyle{definition}
\newtheoremstyle{dotless}{}{}{}{}{\bfseries}{}{ }{}
\theoremstyle{definition}
\newtheorem{exercise}{Exercise}[chapter]
\newtheorem{problem}[exercise]{Problem}
\def\bx{\begin{exercise}}
\def\ex{\end{exercise}}
\def\bp{\begin{problem}}
\def\ep{\end{problem}}

\newtheorem{extrap}{Extra Problem}
\newtheorem{extrae}[extrap]{Extra Exercise}

\theoremstyle{remark}
\newtheorem*{solution}{Solution}

%environment shortcuts 
  \def\ba{\begin{array}}
  \def\ea{\end{array}}
  \def\bc{\begin{corollary}}
  \def\ec{\end{corollary}}
  \def\bd{\begin{definition}}
  \def\ed{\end{definition}}
  \def\ben{\begin{enumerate}}
  \def\een{\end{enumerate}}
  \def\bse{\begin{equation*}}
  \def\ese{\end{equation*}}
  \def\be{\begin{example}}
  \def\ee{\end{example}}
  \def\bi{\begin{IEEEeqnarray*}}
  \def\ei{\end{IEEEeqnarray*}}
  \def\bit{\begin{itemize}}
  \def\eit{\end{itemize}}
  \def\bl{\begin{lemma}}
  \def\el{\end{lemma}}
  \def\bnn{\begin{notation}}
  \def\enn{\end{notation}}
  \def\bn{\begin{note}}
  \def\en{\end{note}}
%  \def\bp{\begin{proposition}}
%  \def\ep{\end{proposition}}
  \def\bq{\begin{proof}}
  \def\eq{\end{proof}}
  \def\br{\begin{remark}}
  \def\er{\end{remark}}
  \def\bs{\begin{solution}}
  \def\es{\end{solution}}
  \def\btab{\begin{table}}
  \def\etab{\end{table}}
  \def\btb{\begin{tabular}}
  \def\etb{\end{tabular}}
  \def\bt{\begin{theorem}}
  \def\et{\end{theorem}}

%miscellaneous shortcuts
  \def\a{\alpha}
  \def\ax{\mathrm{ax}}
  \def\b{\beta}
  \def\C{\mathbb{C}}
  \def\cA{\mathcal{A}}
  \def\cF{\mathcal{F}}
  \def\cH{\mathcal{H}}
  \def\cJ{\mathcal{J}}
  \def\cK{\mathcal{K}}
  \def\cL{\mathcal{L}}
  \def\cO{\mathcal{O}}
  \def\cP{\mathcal{P}}
  \def\cl{\colon}
  \def\co{\mathrm{co}}
  \def\D{\Delta}
  \def\d{\mathrm{d}}
  \def\ds{\displaystyle}
  \def\e{\mathrm{e}}
  \def\eqv{\Leftrightarrow}
  \def\F{\mathbb{F}}
  \def\g{\gamma}
  \def\ic{\mathrm{i}}
  \def\id{\mathrm{id}}
  \def\img{\mathrm{im}}
  \def\imp{\Rightarrow}
  \def\inj{\mathrm{in}}
  \def\iset{\cong_\mathrm{set}}
  \def\l{\lambda}
  \def\la{\langle}
  \def\Mat{\mathrm{Mat}}
  \def\m{\mathrm{m}}
  \def\N{\mathbb{N}}
  \def\n{\nabla}
  \def\ol{\overline}
  \def\op{\mathrm{op}}
  \def\p{\partial}
  \def\pr{\mathrm{pr}}
  \def\Q{\mathbb{Q}}
  \def\R{\mathbb{R}}
  \def\ra{\rangle}
  \def\re{\Re\e}
  \def\S{\Sigma}
  \def\s{\sigma}
  \def\se{\subseteq}
  \def\sm{\setminus}
  \def\ss{\subset}
  \def\t{\text}
  \def\ua{\nearrow}
  \def\ve{\varepsilon}
  \def\vn{\varnothing}
  \def\wto{\rightharpoonup}
  \def\Z{\mathbb{Z}}


\def\lacts{\vartriangleright}
\def\racts{\vartriangleleft}
\def\smallblackbox{\mathbin{\raisebox{0.6pt}{\scalebox{0.55}{$\blacksquare$}}}}

\DeclareMathOperator{\AC}{AC}
\DeclareMathOperator{\Ad}{Ad}
\DeclareMathOperator{\Aut}{Aut}
\DeclareMathOperator{\ad}{ad}
\DeclareMathOperator{\Borel}{Borel}
\DeclareMathOperator{\coker}{coker}
\DeclareMathOperator{\Der}{Der}
\DeclareMathOperator{\Eig}{Eig}
\DeclareMathOperator{\End}{End}
\DeclareMathOperator*{\esup}{ess\,sup}
\DeclareMathOperator{\ev}{ev}
\DeclareMathOperator{\Gr}{Gr}
\DeclareMathOperator{\Hom}{Hom}
\DeclareMathOperator{\Homeo}{Homeo}
%\DeclareMathOperator{\id}{id}
\DeclareMathOperator{\im}{im}
\DeclareMathOperator{\mor}{mor}
\DeclareMathOperator{\ob}{ob}
%\DeclareMathOperator{\pr}{pr}
\DeclareMathOperator{\preim}{preim}
\DeclareMathOperator{\proj}{proj}
\DeclareMathOperator{\ran}{ran}
\DeclareMathOperator{\sgn}{sgn}
\DeclareMathOperator{\lspan}{span}
\DeclareMathOperator{\tr}{tr}
\newcommand{\tvb}[3]{\left(\frac{\partial}{\partial {#1}^{#2}}\right)_{\negmedspace #3}}
\DeclareMathOperator{\vol}{vol}

\DeclareMathOperator*{\slim}{s-lim}
\DeclareMathOperator*{\wlim}{w-lim}


\renewcommand\Re{\operatorname{Re}}
\renewcommand\Im{\operatorname{Im}}

\newcommand{\myparallel}{{\mkern3mu\vphantom{\perp}\vrule depth 0pt\mkern2mu\vrule depth 0pt\mkern3mu}}




\DeclareMathOperator{\GL}{GL}
\def\gl{\mathfrak{gl}}
\DeclareMathOperator{\Ort}{O}
\def\ort{\mathfrak{o}}
\DeclareMathOperator{\SL}{SL}
\def\sl{\mathfrak{sl}}
\DeclareMathOperator{\SO}{SO}
\def\so{\mathfrak{so}}
\DeclareMathOperator{\SU}{SU}
\def\su{\mathfrak{su}}



% \DeclareFontEncoding{LS1}{}{}
% \DeclareFontSubstitution{LS1}{stix}{m}{n}
% \DeclareSymbolFont{symbols2}{LS1}{stixfrak}{m}{n}
% \DeclareMathSymbol{\Lvzigzag}{\mathopen}{symbols2}{"C9}
% \DeclareMathSymbol{\Rvzigzag}{\mathopen}{symbols2}{"CA}







% \DeclareFontFamily{U}{matha}{}
% \DeclareFontShape{U}{matha}{m}{n}{
%   <-5.5>    matha5
%   <5.5-6.5> matha6 
%   <6.5-7.5> matha7
%   <7.5-8.5> matha8
%   <8.5-9.5> matha9
%   <9.5-11>  matha10
%   <11->     matha12
% }{}
% \DeclareSymbolFont{matha}{U}{matha}{m}{n}
% \DeclareFontSubstitution{U}{matha}{m}{n}
% \DeclareFontFamily{U}{mathx}{\hyphenchar\font45}
% \DeclareFontShape{U}{mathx}{m}{n}{<-> mathx10}{}
% \DeclareSymbolFont{mathx}{U}{mathx}{m}{n}
% \DeclareFontSubstitution{U}{mathx}{m}{n}

% \DeclareMathDelimiter{\lfilet}{4}{mathx}{"37}{mathx}{"37}
% \DeclareMathDelimiter{\rfilet}{5}{mathx}{"3F}{mathx}{"3F}







\DeclareFontFamily{U}{MnSymbolC}{}
\DeclareSymbolFont{MnSyC}{U}{MnSymbolC}{m}{n}
\DeclareMathSymbol{\diamondplus}{\mathbin}{MnSyC}{"7C}
\DeclareMathSymbol{\diamonddot}{\mathbin}{MnSyC}{"7E}
\DeclareFontShape{U}{MnSymbolC}{m}{n}{
    <-6>  MnSymbolC5
   <6-7>  MnSymbolC6
   <7-8>  MnSymbolC7
   <8-9>  MnSymbolC8
   <9-10> MnSymbolC9
  <10-12> MnSymbolC10
  <12->   MnSymbolC12}{}


\title{Solutions to Strom's\\\textit{Modern Classical Homotopy Theory}}
\author{Simon Rea}
\date{Last updated: \today}



\begin{document}

\maketitle
\tableofcontents

\part{The Language of Categories}
\chapter{Categories and Functors}

\section{Diagrams}

\bx
Show that $\mathcal{D}$ is commutative if, and only if, $\overline{\mathcal{D}}$ is commutative.
\ex

\bs
The commutativity of $\mathcal{D}$ amounts to the equality between one or more composites $f_n\circ\cdots \circ f_1=g_m\circ \cdots\circ g_1$. The diagram $\overline{\mathcal{D}}$ is formed by adding all the identity morphisms, which obviously do not change the composites, as well as all compositions of morphisms  already present in $\mathcal{D}$. These will be of the form $\widetilde f=f_{i+1}\circ f_i$ and thus, by associativity, also leave the above composites unchanged.
\es

\bx
\ben[label=(\alph*)]
\item Take some time to convince yourself that the given definition of algebraic closure actually does define what you think of as algebraic closure.
\item The isomorphism theorems of elementary group theory can be written down in terms of diagrams. Do it!
\item Rewrite the statement that the cube-shaped diagram 
\bse
\begin{tikzcd}
X_1 \ar[rr,"f_{12}"] \ar[dr,"f_{13}"] \ar[dd,"h_1"'] && X_2 \ar[dr,"f_{24}"] \ar[d,shorten >= 1ex,dash]&\\
& X_3\ar[rr,"f_{34}" near start] \ar[dd,"h_3"' near start] & \ar[d,"h_2"] & X_4 \ar[dd,"h_4"]\\
Y_1 \ar[dr,"g_{13}"] \ar[r,dash] & \ar[r,"g_{12}"] & Y_2 \ar[dr,"g_{24}"] &\\
& Y_3 \ar[rr,"g_{34}"] && Y_4
\end{tikzcd}
\ese
is commutative without using any diagrams at all.
\een
\ex

\bs
\ben[label=(\alph*)]
\item This statement appears to be incorrect\footnote{See \url{https://math.stackexchange.com/q/1033209} for further details.}.
\item The first isomorphism theorem can be restated as follows. Every group homomorphism $\varphi\cl G\to H$ can be factored as
\bse
\begin{tikzcd}[column sep=large]
G \ar[dr,bend right=30,"q"']\ar[rrr,"\varphi"] &&& H\\
& G/\ker(\varphi) \ar[r,"\widetilde\varphi"] & \im(\varphi) \ar[ur,bend right=30,hook,"i"'] &
\end{tikzcd}
\ese
where $q$ is the quotient homomorphism, $\widetilde \varphi$ is an isomorphism, and $i$ is the inclusion homomorphism. Looking ahead at Section 1.5, we can further restate this as a collection of lifting and extension problems only. Indeed, here we are looking for an extension $\epsilon$
\bse
\begin{tikzcd}[row sep=large,column sep=huge]
&\ar[dl,bend right=10,"\epsilon"']G/\ker(\varphi)\\
H & \ar[l,"\varphi"]G\ar[u,"q"']
\end{tikzcd}
\ese
such that there is a lifting $\lambda$
\bse
\begin{tikzcd}[row sep=large,column sep=huge]
&\im(\varphi)\ar[d,hook,"i"]\\
\ar[ur,bend left=10,"\lambda"] G/\ker(\varphi) \ar[r,"\epsilon"']& H.
\end{tikzcd}
\ese
The condition that $\lambda$ be an isomorphism can, in turn, be expressed in terms of an extension and a lifting problem as in the footnote exercise in Section 1.5.

The second isomorphism theorem states that... Putting all the homomorphisms in one diagram, we have
\bse
\begin{tikzcd}[row sep=large,column sep=large]
G/N \ar[rrr,bend left=30,"\cong"]& \ar[l,"q"']G\ar[r,"q"] & G/H \ar[r,"q"]& {\displaystyle\frac{G/H}{N/H}}\\
& N \ar[ul,"*"]\ar[u,hook]\ar[r,"q"]& N/H\ar[u,hook] \ar[ur,"*"']&\\
& H \ar[u,hook]\ar[ur,"*"']&&
\end{tikzcd}
\ese
where the $q$'s are quotient homomorphisms, the hooked arrows are inclusions of subgroups, and $*$ indicates a trivial homomorphism.
\item By associativity, it suffices to only consider composites of up to two morphisms.
\begin{table}[h!]
\def\arraystretch{1.25}
\centering
\begin{tabular}{c||c}
\ Cube face\ \ & Condition\\
\hline
\hline
Top & $f_{24}\circ f_{12}=f_{34}\circ f_{13}$ \\
Back & $h_{2}\circ f_{12}=g_{12}\circ h_{1}$\\
Left & $h_{3}\circ f_{13}=g_{13}\circ h_{1}$\\
Right & $h_{4}\circ f_{24}=g_{24}\circ h_{2}$\\
Front & $h_{4}\circ f_{34}=g_{34}\circ h_{3}$\\
Bottom & $g_{24}\circ g_{12}=g_{34}\circ g_{13}$
\end{tabular}
\end{table}
\een
\es

\section{Categories}

\bx
\ben[label=(\alph*)]
\item Give five examples of categories besides the ones already mentioned.
\item Find a way to interpret a group $G$ as a category with a single object.
\item Let $X$ be a topological space. Show how to make a category whose objects are the points of $X$ and such that the set of morphisms from $a$ to $b$ is the set of all paths $\omega \cl [0, d] \to X$ (where $d \geq 0$) such that $\omega(0) = a$ and $\omega(d) = b$.
\item Suppose $\mathcal{D}$ is a diagram in the sense of Section 1.1. Show that $\overline{\mathcal{D}}$ is a category.
\een
\ex

\bs
\ben[label=(\alph*)]
\item 
\ben[label=(\roman*)]
\item The category $\mathbf{Sets}$ of sets and functions between them.
\item Given a unital ring $R$, we have the category $R$-$\mathbf{Mod}$ of left $R$-modules and $R$-module homomorphisms. Similarly, we also have the categories of right $R$-modules $\mathbf{Mod}$-$R$ and $R$-$S$-bimodules $R$-$\mathbf{Mod}$-$S$, for some other unital ring $S$.

As a special case, if $\F$ is a field, then $\F$-$\mathbf{Mod}$ is the category of vector spaces over $\F$ and $\F$-linear maps, sometimes denoted as $\mathbf{Vect}_{\F}$. 
\item Given a class $S$, we can form a category with objects the elements of $S$ and only identity morphisms. It is called the \emph{discrete category} with object class $S$. This includes the finite categories $\mathbf{1}, \mathbf{2},\ldots, \mathbf{n},\ldots,$ with objects the positive integers up to $1,2,\ldots,n,\ldots,$ respectively. Of course, we also have the empty category $\vn$ with no objects and no morphisms.  
\item Given a unital ring $R$, the category $\mathbf{Mat}(R)$ of matrices with entries in $R$ has the natural numbers $\N$ as objects, and for $n,m\in \mathbf{Mat}(R)$, the set of morphisms $\mor_{\mathbf{Mat}(R)}(n,m)$ is the set $\Mat_{m\times n}(R)$ of $m\times n$ matrices with entries in $R$.

The composite $B\circ A\cl n\to p$ of $A\cl n\to m$ and $B\cl m\to p$ is the product matrix $BA\in\Mat_{p\times n}(R)$ and the identity $\id_n\cl n\to n$ is just $I_n\in \Mat_{n\times n}(R)$. That these make $\mathbf{Mat}(R)$ into a category follows at once from the properties of matrix multiplication.

This is one of few examples in which a category is named after its morphisms rather than its objects.
\item The category $\mathbf{hTop}$, whose objects are topological spaces and whose morphisms are homotopy classes of continuous maps. The identity $\id_X$ in $\mathbf{hTop}$ is just $[\id_X]$, with $\id_X$ in $\mathbf{Top}$, and composition is defined by $[g]\circ [f]=[g\circ f]$. This is an example of a \emph{quotient category}.

In general, given a category $\mathcal{C}$, a \emph{congruence relation} $\sim$ on $\mathcal{C}$ consists in an equivalence relation $\sim_{X,Y}$ on $\mor_{\mathcal{C}}(X,Y)$ for each pair of objects $X$ and $Y$ which respects composition. That is, given $f_1,f_2\in\mor_{\mathcal{C}}(X,Y)$ and $g_1,g_2\in\mor_{\mathcal{C}}(Y,Z)$ such that $f_1\sim_{X,Y}f_2$ and $g_1\sim_{Y,Z}g_2$, we require that $g_1\circ f_1\sim_{X,Z}g_2\circ f_2$. We can then define $\mathcal{C}/{\sim}$ by setting $\ob(\mathcal{C}/{\sim})=\ob(\mathcal{C})$ and $\mor_{\mathcal{C}/{\sim}}(X,Y)=\mor_{\mathcal{C}}(X,Y)/{\sim_{X,Y}}$. 

One can check that the homotopy relation $\simeq$ is a congruence relation on $\mathbf{Top}$ and that $\mathbf{hTop}=\mathbf{Top}/{\simeq}$.
\een
\item Given a group $G$, define a category $\mathcal{G}$ by setting $\ob(\mathcal{G})=\{*\}$. Then $\mathcal{G}$ has a single morphism set and we can declare $\mor_{\mathcal{G}}(*,*)=G$ and define composition of morphisms in $\mor_{\mathcal{G}}(*,*)$ to be the binary operation of $G$. The associativity and identity axioms of $G$ imply that $\mathcal{G}$ is a category. 

The same construction works just as well even if $G$ is just a monoid. The existence of inverses in the group $G$ implies that every morphism in $\mor_{\mathcal{G}}(*,*)$ is an equivalence.

Vice versa, given a category $\mathcal{C}$ with only one object $*$, the axioms in the definition of a category imply that $\mor_{\mathcal{C}}(*,*)$ is a monoid under composition of morphisms. If, in addition, every morphism in $\mor_{\mathcal{C}}(*,*)$ is an equivalence, then $\mor_{\mathcal{C}}(*,*)$ is a group under $\circ$.

So monoids and groups are exactly the same as categories with one object and categories with one object in which every morphism is an equivalence, respectively.
\item Let $a,b,c\in X$ and let $\omega\cl a\to b$ and $\gamma\cl b\to c$ be paths in $X$ with $\omega\cl[0,d_1]\to X$ and $\gamma\cl[0,d_2]\to X$, respectively. We define the composite $\gamma\circ \omega\cl a\to c$ to be $\omega * \gamma  \cl [0,d_1+d_2] \to  X$, where 
\bse
(\omega*\gamma) (t) = \begin{cases}\omega(t), &0\leq t\leq d_1\\ \gamma(t-d_1), & d_1\leq t \leq d_1+d_2 .\end{cases}
\ese
For any $a\in X$, we define $\id_a\cl a \to a$ by $\id_a\cl [0,0]\to X$, with $\id_a(0)=a$. Then $\omega\circ \id_a=\omega=\id_b\circ\omega$ for any $\omega\cl a\to b$.

Finally, given $\omega\cl a\to b$, $\gamma\cl b\to c$ and $\eta\cl c\to d$, we have
\bi{rCl}
\eta\circ(\gamma\circ \omega) & = & (\omega * \gamma)*\eta\\
& = & \left( t\mapsto \begin{cases}\omega(t), & 0\leq t \leq d_1\\ \gamma(t-d_1), & d_1\leq t\leq d_1+d_2\end{cases} \right)\! * \eta\\
& = & t\mapsto \begin{cases}\omega(t), & 0\leq t \leq d_1\\ \gamma(t-d_1), & d_1\leq t\leq d_1+d_2\\ \eta(t-d_1-d_2),&d_1+d_2\leq t\leq d_1+d_2+d_3\end{cases}
\ei
and
\bi{rCl}
(\eta\circ\gamma)\circ \omega & = & \omega * (\gamma*\eta)\\
& = & \omega * \! \left( t\mapsto \begin{cases}\gamma(t), & 0\leq t \leq d_2\\ \eta(t-d_2), & d_2\leq t\leq d_2+d_3\end{cases} \right)\\
& = & t\mapsto \begin{cases}\omega(t), & 0\leq t \leq d_1\\ \gamma(t-d_1), & d_1\leq t\leq d_2+d_1\\ \eta(t-d_2-d_1),&d_2+d_1\leq t\leq d_2+d_3+d_1,\end{cases}
\ei
so $\eta\circ(\gamma\circ \omega) =(\eta\circ\gamma)\circ \omega$.
\item By construction, $\overline{\mathcal{D}}$ contains the identity morphism of each of its objects, as well as the composite of every pair of morphisms. The associativity axiom follows from the fact that $\overline{\mathcal{D}}$ is commutative.
\een
\es

\bx
Show that if such a $g$ exists, it is unique.
\ex

\bs
Suppose that we also have $h\cl Y \to X$ such that $h\circ f=\id_X$ and $f\circ h=\id_Y$. Then
\bse
h = h \circ \id_Y=h\circ (f\circ g) = (h\circ f)\circ g =\id_X \circ g = g.
\ese
\es

\bp
Let’s say $X \sim Y$ if there is an equivalence $f \cl X \to Y$.
\ben[label=(\alph*)]
\item Show that $\sim$ is an equivalence relation.
\item Interpret `equivalence' in each of the categories that have been discussed in the text so far, including the ones you found in Exercise 1.3.
\een 
\ep

\bs
\ben[label=(\alph*)]
\item The identity $\id_X$ is an equivalence, so $X\sim X$. If $f \cl X \to Y$ is an equivalence with inverse $f^{-1}$, then $f^{-1}\cl Y \to X$ is an equivalence with inverse $f$. So $X\sim Y$ implies $Y\sim X$. Finally, given equivalences $f \cl X \to Y$ and $g \cl Y \to Z$, we have
\bse
(g\circ f)\circ (f^{-1}\circ g^{-1})=g\circ (f\circ f^{-1})\circ g^{-1}=g\circ g^{-1}=\id_Z
\ese
and
\bse
(f^{-1}\circ g^{-1})\circ (g\circ f)= f^{-1} \circ (g^{-1}\circ g) \circ f=f^{-1} \circ f=\id_X.
\ese
So $g\circ f\cl X\to Z$ is an equivalence with inverse $(g\circ f)^{-1}=f^{-1}\circ g^{-1}$. Thus $X\sim Y$ and $Y\sim Z$ imply $X\sim Z$.
\item 
\ben[label=(\roman*)]
\item In $\mathbf{Sets}$, the equivalences are precisely the bijections of sets. To see this, first note that the unique morphism $\vn \to \vn$ is an equivalence and that, if $A$ is non-empty, then there is no morphism $A\to\vn$, so the unique morphism $\vn \to A$ cannot be an equivalence (see Exercise 1.43 for more details). Thus, $\vn$ is equivalent only to itself.

Hence, suppose that $A$ and $X$ are non-empty. Then, if $f\cl A \to X$ is a bijection, its inverse $f^{-1}\cl X \to A$ is such that $f^{-1}\circ f=\id_A$ and $f\circ f^{-1}=\id_X$. Conversely, let $f\cl A \to X$ and suppose that there exists $g\cl X \to A$ such that $g\circ f=\id_A$ and $f\circ g=\id_X$. Let $a,b\in A$ be such that $f(a)=f(b)$. Since $g\circ f=\id_A$, we have $g(f(a))=g(f(b))$, hence $\id_A(a)=\id_A(b)$, so $a=b$ and $f$ is injective. Since $f\circ g=\id_X$, for every $x\in X$, we have $x=f(g(x))$. Thus, for all $x\in X$, there exists $a\in A$, namely $a=g(x)$, such that $f(a)=x$. So $f$ is surjective. Hence $f$ is a bijection.

In fact, more is true. A morphism $f\cl A \to X$ is injective if, and only if, there exists a $g\cl X \to A$ such that $g\circ f=\id_A$. Indeed, if $f$ is injective, then for all $x\in X$, the preimage $f^{-1}(\{x\})$ contains at most one element. We define $g\cl X\to A$ by
\bse
g(x)=\begin{cases}a & \text{ if }a\in f^{-1}(\{x\})\\ a_0 & \text{ if }f^{-1}(\{x\})=\vn,\end{cases}
\ese
where $a_0$ is any fixed element of $A$. Then, for each $a\in A$, we have $f^{-1}(\{f(a)\})=\{a\}$, so $(g\circ f)(a)=a$ for all $a\in A$, and thus $g\circ f=\id_A$.

Similarly, $f\cl A \to X$ is surjective if, and only if, there exists a $g\cl X \to A$ such that $f\circ g=\id_X$. Indeed, if $f$ is surjective, for each $x\in X$, the preimage $f^{-1}(\{x\})$ is non-empty. Thus (invoking the axiom of choice) we can define $g\cl X\to A$ by mapping each $x$ to any one element of $f^{-1}(\{x\})$ (i.e.\ we let $g$ be a choice function for the family $\{f^{-1}(\{x\})\mid x\in X\}$). Then, for each $x\in X$, we have $g(x)\in f^{-1}(\{x\})$ and hence $f(g(x))=x$ for all $x\in X$. Thus, $f\circ g=\id_X$.

%Therefore, a morphism in $\mathbf{Sets}$ is an equivalence in the sense of category theory if, and only if, it is a bijection. 
\item The $R$-module isomorphisms clearly satisfy the definition of equivalence in $R$-$\mathbf{Mod}$. Recall that, if $f\cl M\to N$ is a bijective $R$-module homomorphism, then its inverse $f^{-1}$ (as a map of sets) is also an $R$-module homomorphism. Indeed, given $r\in R$ and $m_1,m_2\in M$, we have
\bi{rCl}
f(r\cdot f^{-1}(m_1)+f^{-1}(m_2)) & = & r\cdot f( f^{-1}(m_1))+f(f^{-1}(m_2))\\
& = & r \cdot m_1 + m_2\\
& = & f( f^{-1}(r \cdot m_1 + m_2))
\ei
and thus, since $f$ is injective, $f^{-1}(r \cdot m_1 + m_2)=r\cdot f^{-1}(m_1)+f^{-1}(m_2)$.
This, together with part (i), implies that the equivalences in $R$-$\mathbf{Mod}$ are precisely the $R$-module isomorphisms.

The situation is completely analogous with $\mathbf{Mod}$-$R$, $R$-$\mathbf{Mod}$-$S$ and $\mathbf{Vect}_{\F}$, as well as $\mathbf{Grp}$ and $\mathbf{AbGrp}=\Z$-$\mathbf{Mod}$ (see Exercise A.1 for this equality).

\item In a discrete category, the only morphisms are the identity morphisms, and they are all equivalences. The converse is not true, however. That is, a category whose morphisms are all equivalences need not be a discrete category. Such categories are called \emph{groupoids}, and a groupoid with only a single object is a group in the sense of Exercise 1.3 (b).
\item A morphism $A\cl n\to m$ in $\mathbf{Mat}(R)$ is an equivalence if there exists $B\cl m\to n$ such that $BA=I_n$ and $AB=I_m$. If $R$ is commutative, then the trace operator (the sum of the diagonal entries) satisfies $\tr(XY)=\tr(YX)$ for any $X,Y\in \mathbf{Mat}(R)$ for which both $XY$ and $YX$ exist. Then, we have
\bse
m=\tr(I_m)=\tr(AB)=\tr(BA)=\tr(I_n)=n,
\ese
so every equivalence in $\mathbf{Mat}(R)$ with $R$ commutative must be an \emph{automorphism}, i.e.\ an invertible morphism with the same domain and target.
\item The equivalences in $\mathbf{Top}$ are continuous bijections whose inverses are also continuous, i.e.\ homeomorphisms. Note that, unlike the algebraic categories considered above, a continuous bijection is not necessarily a homeomorphism.

The equivalences in $\mathbf{hTop}$ are (homotopy classes of) homotopy equivalences, i.e.\ $[f]\cl X\to Y$ is an equivalence in $\mathbf{hTop}$ if there exists $g\cl Y \to X$ in $\mathbf{Top}$ such that $g\circ f$ is homotopic to $\id_X$ and $f\circ g$ is homotopic to $\id_Y$.
\een
The equivalences in a monoid $M$ seen as a category, as in Exercise 1.3 (b), are simply the invertible elements of $M$, while in a groupoid every morphism is (by definition) an equivalence.

Finally, the category with objects $\N$ and morphisms given by the divisibility relation, that with objects $\R$ and morphisms given by the less-than-or-equal relation, and that described in Exercise 1.3 (c) are all examples of \emph{skeletal categories}, i.e.\ categories in which the only equivalences are the identity morphisms. Indeed, if $n,m>0$, then $n\mid m$ and $m\mid n$ if, and only if, $n=m$; clearly $x\leq y$ and $y\leq x$ if, and only if, $x=y$ in $\R$; and finally, given $\omega\cl [0,d]\to X$ with $d>0$, it is clear that there exists no path $\gamma\cl [0,d']$ such that $\omega * \gamma$ or $\gamma * \omega$ is the identity, as this would require $d+d'=0$.
\een
\es

\bp
Let $\mathcal{C}$ be a category and let $f \cl X \to Y$ be a map which has a left inverse $g \cl Y \to X$, and suppose $g$ also has a left inverse. Show that $f$ and $g$ are two-sided inverses of each other.
\ep

\bs
Let $h\cl X\to Y$ be a left inverse to $g$. Then $g\circ f=\id_X$ and $h\circ g=\id_Y$. We have
\bse
f = \id_Y\circ f=(h\circ g)\circ f=h\circ (g\circ f)=h\circ \id_X=h,
\ese
so $g\circ f=\id_X$ and $f\circ g=\id_Y$.
\es


\bx
Whenever we use the term `retract', we should be referring to the definition above, where an object $A$ was a retract of another object $X$ in some category. By setting up an appropriate category, show that our definition of $f$ being a retract of $g$ can be thought of as an instance of that general categorical definition. What is does it mean, in terms of the category $\mathcal{C}$, for two objects to be equivalent in your new category?

{\scshape Hint}. Obviously, $f$ and $g$ must be among the objects in your category!
\ex

\bs
Fix a category $\mathcal{C}$ and let $\mathcal{M}$ be the category with objects the morphisms of $\mathcal{C}$, so $f\in \ob(\mathcal{M})$ if $f\in \mor_{\mathcal{C}}(A,B)$ for some $A,B\in \ob(\mathcal{C})$.

A morphism from $A\xrightarrow{f}B$ to $X\xrightarrow{g}Y$ is a pair of morphisms $(A\xrightarrow{i}X,B\xrightarrow{j}Y)$ in $\mathcal{C}$ such that the square
\bse
\begin{tikzcd}[row sep=large,column sep=large]
A \ar[d,"f"]\ar[r,"i"]& X\ar[d,"g"]\\
B\ar[r,"j"] & Y
\end{tikzcd}
\ese
commutes, i.e.\ $j\circ f=g\circ i$.

If $f\cl A\to B$, the identity $\id_f\cl f\to f$ is given by $\id_f=(\id_A,\id_B)$ and if $(i,j)\cl f\to g$ and $(k,l)\cl g\to h$ are morphisms in $\mathcal{M}$, their composite is defined by
\bse
(k,l)\circ (i,j)=(k\circ i,l\circ j) \cl f\to h.
\ese
With these definitions, the associativity and identity laws in $\mathcal{M}$ follow directly from those of $\mathcal{C}$.

The categorical definition of a retract in $\mathcal{M}$ reads as: `$f$ is a retract of $g$ if there is a commutative diagram'
\bse
\begin{tikzcd}[row sep=large,column sep=large]
f \ar[rr,bend left=45,"\id_f"]\ar[r,"{(i,j)}"'] & g \ar[r,"{(r,s)}"'] & f 
\end{tikzcd}
\ese
Writing all morphisms explicitly, if $f\cl A\to B$ and $g\cl X\to Y$, we have
\bse
\begin{tikzcd}[row sep=large,column sep=large]
A \ar[d,"f"]\ar[rr,bend left=45,"\id_A"]\ar[r,"i"] & X \ar[r,"r"] \ar[d,"g"]& A\ar[d,"f"]\\
B \ar[r,"j"] \ar[rr,bend right=45,"\id_B"] & Y\ar[r,"s"] & B
\end{tikzcd}
\ese
So the second definition given in the text is, in fact, an instance of the general categorical definition of a retract.

Finally, note that for $f\cl A\to B$ and $g\cl X\to Y$ to be equivalent in $\mathcal{M}$ it means that $A$ and $B$ are equivalent to $X$ and $Y$ via equivalences $i$ and $j$, respectively, which commute with $f$ and $g$.
\es

\bx
Find examples of retracts in algebra, topology, and other contexts.
\ex

\bs
In the category $\mathbf{Grp}$, the trivial group $\{e\}$ is a retract of every other group as, for any group $G$, there are (unique) homomorphisms $\{e\} \to G \to \{e\}$, which compose to the identity on $\{e\}$. In general, a group $H$ is a retract of $G$ if, and only if, $H$ is (isomorphic to) a subgroup of $G$ which has a normal complement, i.e.\ a normal subgroup $K$ of $G$ such that $H\cap K=\{e\}$ and, for all $g\in G$, $g=hk$ for some $h\in H$, $k\in K$. Then $K$ is the kernel of the retraction $G\to H$.

In the category $\mathbf{Top}$, any one-point space $P$ is a retract of every other space. Indeed, for any space $X$, there are (unique) maps $X\to P$ and $P\to P$, so that $P\to X\to P$ is $\id_P$ for all maps $P\to X$. Retracts of a space retain some properties of that space, such as connectedness, but the converse is often false. There are several variants of this concept in topology, for instance that of a `deformation retraction', where in addition to $A\to X\to A$ being the identity on $A$, we also require that $X\to A\to X$ be homotopic to $\id_X$. A space $A$ is a retract of $X$ in $\mathbf{hTop}$ if $A$ is a `weak retract' of $X$ in $\mathbf{Top}$, i.e.\ if $A\to X\to A$ is homotopic to $\id_A$. 

In the category $\mathbf{Man}^{\infty}$ of smooth manifolds and smooth maps, if a smooth bundle $E\xrightarrow{\,\pi\,}M$ admits a global section $\sigma\cl M\to E$, then $M$ is a retract of $E$.

As a final note, we observe that if $\mathcal{C}$ has a terminal object $\tau$, then $\tau$ is a retract of every object in $\mathcal{C}$, as was the case for $\{e\}\in\mathbf{Grp}$ and $P\in\mathbf{Top}$.
\es

\bp
Let $f \cl A \to B$ and $g \cl X \to Y$ be morphisms in a category $\mathcal{C}$. Assume that $f$ is a retract of $g$.
\ben[label=(\alph*)]
\item Show that if $g$ is an equivalence, then $f$ is also an equivalence.
\item Show by example that $f$ can be an equivalence even if $g$ is not an equivalence.
\een
\ep

\bs
\ben[label=(\alph*)]
\item Referring to the diagram in the solution to Exercise 1.7, consider the morphism $r\circ g^{-1}\circ j\cl B\to A$. We have
\bse
f\circ (r\circ g^{-1}\circ j) = (f\circ r)\circ g^{-1}\circ j= s\circ g\circ g^{-1}\circ j =s\circ j=\id_B
\ese
and
\bse
(r\circ g^{-1}\circ j)\circ f = r\circ g^{-1}\circ (j\circ f) = r\circ g^{-1}\circ g\circ i = r\circ i=\id_A.
\ese
So $f$ is an equivalence with inverse $f^{-1}=r\circ g^{-1}\circ j$.
\item
\een
\es

\section{Functors}

\bx
\ben[label=(\alph*)]
\item  Let $\mathcal{D}$ be a diagram in a category $\mathcal{A}$, and let $\overline{\mathcal{D}}$ be the category obtained
from it as in Exercise 1.3(d). Show that $\mathcal{D}$ is commutative if, and only if, for any two objects $X, Y \in \mathcal{D}$, $\mor_{\overline{\mathcal{D}}}(X, Y )$ has at most one element.
\item Show that if you apply a covariant functor $F \cl \mathcal{A} \to \mathcal{B}$ to a commutative diagram in $\mathcal{A}$, the result is a commutative diagram in $\mathcal{B}$.
\een
\ex

\bs
\ben[label=(\alph*)]
\item Assuming that there is at most one morphism between any two objects in $\mathcal{D}$ to begin with, the diagram $\overline{\mathcal{D}}$ is formed by adding any identity morphisms not already present in $\mathcal{D}$, so $|\mor_{\overline{\mathcal{D}}}(X,X)|=1$ for each $X$ in $\overline{\mathcal{D}}$, and adding a morphism $X\to Y$ if there is no morphism $X\to Y$ in $\mathcal{D}$ but there are morphisms $f_1,\ldots,f_n$ in $\mathcal{D}$ such that $f_n\circ\cdots\circ f_1\cl X\to Y$. Since $\mathcal{D}$ is commutative, all compositions resulting in a morphism $X\to Y$ are equal, so $|\mor_{\overline{\mathcal{D}}}(X,Y)|\leq 1$.

Conversely, suppose that $|\mor_{\overline{\mathcal{D}}}(X,Y)|\leq 1$ for all $X$ and $Y$ in $\mathcal{D}$. If there are morphisms $f_1,\ldots,f_n$ in $\mathcal{D}$ such that $f_n\circ\cdots\circ f_1\cl X\to Y$, then $f_n\circ\cdots\circ f_1\in\mor_{\overline{\mathcal{D}}}(X,Y)$, so $|\mor_{\overline{\mathcal{D}}}(X,Y)|=1$. Given any other composite $g_m\circ\cdots\circ g_1\cl X\to Y$ of morphisms in $\mathcal{D}$, the fact that $|\mor_{\overline{\mathcal{D}}}(X,Y)|=1$ implies that $f_n\circ\cdots\circ f_1=g_m\circ\cdots\circ g_1$, so $\mathcal{D}$ is commutative.
\item Given a diagram in $\mathcal{A}$, the corresponding diagram in $\mathcal{B}$ has (not necessarily all distinct) objects $FX$ and morphisms $F(f)\cl FX\to FY$, with $X$, $Y$ and $f$ from the diagram in $\mathcal{A}$.

The commutativity of the diagram in $\mathcal{A}$ amounts to the equality of one or more composites $f_n\circ \cdots \circ f_1=g_m\circ \cdots \circ g_1\cl X\to Y$ between two objects $X,Y$ of the diagram. By functoriality, we have
\bse
F(f_n)\circ \cdots \circ F(f_1)=F(g_m)\circ \cdots \circ F(g_1)\cl FX\to FY,
\ese
so the corresponding diagram in $\mathcal{B}$ is also commutative.

Using the definition of an $\mathcal{I}$-shaped diagram in $\mathcal{A}$, given in Section 2.1, as simply a covariant functor $D\cl\mathcal{I}\to\mathcal{A}$, the statement of the exercise is implied by the fact that the composite $F\circ D\cl\mathcal{I}\to\mathcal{B}$ is again a covariant functor. This is shown in Exercise 1.15. 
\een
\es

\bp
Let $F \cl \mathcal{C} \to \mathcal{D}$ be a functor. Show that if $f \cl X \to Y$ is an equivalence in $\mathcal{C}$, then $F(f)$ is an equivalence in $\mathcal{D}$. Is it possible for $F(f)$ to be an equivalence without $f$ being an equivalence?
\ep

\bs
Let $f^{-1}\cl Y\to X$ be the inverse to $f$. Then $F(f^{-1})\cl FY\to FX$, so we can compose it with $F(f)\cl FX\to FY$. By the definition of functor, we have
\bse
F(f^{-1})\circ F(f)=F(f^{-1}\circ f)=F(\id_X)=\id_{FX}
\ese
and
\bse
F(f)\circ F(f^{-1})=F(f\circ f^{-1})=F(\id_Y)=\id_{FY}.
\ese
So $F(f)$ is an equivalence in $\mathcal{D}$ and, moreover, $F(f^{-1})=F(f)^{-1}$.

Consider the finite category with two objects $a,b$ and only one non-identity morphism $f\cl a\to b$. Since there are no morphisms $b\to a$, $f$ is not an equivalence. The (unique) functor into the category $\mathbf{1}$ with a single object and a single morphism (namely, the identity on that object) maps $f$ to an equivalence.
\es

\bx
Let $G$ be a group, and think of it as a category with one object, as in Exercise 1.3. Interpret functors $F \cl G \to \mathbf{Sets}$ in terms of familiar concepts in algebra.
\ex

\bs
A functor $F \cl G \to \mathbf{Sets}$ consists of a map $\{*\}\to \ob(\mathbf{Sets})$, i.e.\ a choice of a set $F(*)=X$, together with a map $F(g)\cl X\to X$ for each $g\in G$ such that $F(e)=\id_X$ (where $e$ is the identity of $G$) and $F(g\cdot h)=F(g)\circ F(h)$ for all $g,h\in G$. In other words, $F$ is the same as a $G$-set.

Explicitly, we get an action of $G$ on $X$ by defining 
\bi{rCl}
G\times X & \to & X\\
(g,x) & \mapsto & F(g)(x).
\ei
\es

\bp
Let $i \cl S^1 \hookrightarrow D^2$ be the inclusion of the circle into the disk. An early success of algebraic topology concerned the existence of a continuous function $r \cl D^2 \to S^1$ such that the composite $r \circ i \cl S^1 \to S^1$ is the identity (in other words: is $S^1$ a retract of $D^2$?). The fundamental group is a covariant functor
\bse
\pi_1 \cl \mathbf{Top} \longrightarrow \mathbf{Grp}
\ese
from the category $\mathbf{Top}$ of topological spaces and continuous functions to the category $\mathbf{Grp}$ of groups and homomorphisms. You may have seen, and we will show later, that $\pi_1(S^1) = \Z$ and $\pi_1(D^2) = 0$. Using this functor, show that there can be no such function $r$.
\ep

\bs
Suppose, for the sake of contradiction, that such an $r\cl D^2\to S^1$ exists. Then $r\circ i=\id_{S^1}$, so $F(r)\circ F(i)=F(r\circ i)=F(\id_{S^1})=\id_{\Z}$. Thus we have the following commutative diagram in $\mathbf{Grp}$.
\bse
\begin{tikzcd}[column sep=large]
\Z \ar[rr,bend left=40,"\id_{\Z}"]\ar[r,"F(i)"']&0\ar[r,"F(r)"']&\Z
\end{tikzcd}
\ese
However, since $0$ is a zero object (see Exercise 1.43) in $\mathbf{Grp}$, the homomorphisms $F(i)$ and $F(r)$ are both trivial and hence $F(r)\circ F(i)$ cannot be the identity.
\es

\bx
Let $F \cl \mathcal{C} \to \mathcal{D}$ be a contravariant functor.
\ben[label=(\alph*)]
\item Show that if you apply $F$ to a commutative diagram in $\mathcal{C}$, the result is a commutative diagram in $\mathcal{D}$.
\item Show that if $f \cl X \to Y$ is an equivalence in $\mathcal{C}$, then $F(f)$ is an equivalence in $\mathcal{D}$.
\een
\ex

\bs
\ben[label=(\alph*)]
\item Given a diagram in $\mathcal{C}$, the corresponding diagram in $\mathcal{D}$ has (not necessarily all distinct) objects $FX$ and morphisms $F(f)\cl FY\to FX$, with $X$, $Y$ and $f$ from the diagram in $\mathcal{C}$.

The commutativity of the diagram in $\mathcal{C}$ amounts to the equality of one or more composites $f_n\circ \cdots \circ f_1=g_m\circ \cdots \circ g_1\cl X\to Y$ between two objects $X,Y$ of the diagram. By functoriality, we have
\bse
F(f_1)\circ \cdots \circ F(f_n)=F(g_1)\circ \cdots \circ F(g_m)\cl FY\to FX,
\ese
so the corresponding diagram in $\mathcal{D}$ is also commutative.
\item Let $f^{-1}\cl Y\to X$ be the inverse to $f$. Then $F(f^{-1})\cl FX\to FY$, so we can compose it with $F(f)\cl FY\to FX$. By the definition of functor, we have
\bse
F(f^{-1})\circ F(f)=F(f\circ f^{-1})=F(\id_Y)=\id_{FY}
\ese
and
\bse
F(f)\circ F(f^{-1})=F(f^{-1}\circ f)=F(\id_X)=\id_{FX}.
\ese
So $F(f)$ is an equivalence in $\mathcal{D}$ and, moreover, $F(f^{-1})=F(f)^{-1}$
\een
\es

\bx
Show that the composite of two functors is a functor. What happens if one or both of the functors is contravariant?
\ex

\bs
Suppose first that $F\cl\mathcal{A}\to\mathcal{B}$ and $G\cl\mathcal{B}\to\mathcal{C}$ are both covariant. Define $G\circ F\cl\mathcal{A}\to \mathcal{C}$ by setting $(G\circ F)A=G(FA)$ on objects and 
\bse
(G\circ F)(f\cl A \to B)=G(F(f)\cl FA\to FB)=GF(f)\cl GFA\to GFB
\ese
on morphisms. As $F$ and $G$ are functors, we have
\bse
(G\circ F)(\id_A)=G(F(\id_A))=G(\id_{FA})=\id_{GFA}
\ese
for all $A\in\mathcal{A}$ and, since they are both covariant,
\bse
(G\circ F)(g\circ f)=G(F(g\circ f))=G(F(g)\circ F(f))=G(F(g))\circ G(F(f))
\ese
for all morphisms $f,g$ in $\mathcal{A}$. So $G\circ F$ is again a covariant functor.

If $F$ is covariant and $G$ is contravariant, we have
\bse
(G\circ F)(f\cl A \to B)=G(F(f)\cl FA\to FB)=GF(f)\cl GFB\to GFA
\ese
and
\bse
(G\circ F)(g\circ f)=G(F(g\circ f))=G(F(g)\circ F(f))=G(F(f))\circ G(F(g))
\ese
while, if $F$ is contravariant and $G$ is covariant, we have
\bse
(G\circ F)(f\cl A \to B)=G(F(f)\cl FB\to FA)=GF(f)\cl GFB\to GFA
\ese
and
\bse
(G\circ F)(g\circ f)=G(F(g\circ f))=G(F(f)\circ F(g))=G(F(f))\circ G(F(g)).
\ese
Finally, if $F$ and $G$ are both contravariant, we have
\bse
(G\circ F)(f\cl A \to B)=G(F(f)\cl FB\to FA)=GF(f)\cl GFA\to GFB
\ese
and
\bse
(G\circ F)(g\circ f)=G(F(g\circ f))=G(F(f)\circ F(g))=G(F(g))\circ G(F(f)).
\ese
Thus, we see that $G\circ F$ is covariant if $F$ and $G$ have the same variance, and contravariant otherwise.
\es

\bx
Can there be a functor $F \cl \mathcal{C} \to \mathcal{D}$ that is both covariant and contravariant? What special properties must such a functor have?
\ex

\bs
Given $f\cl X\to Y$ in $\mathcal{C}$, we require that $F(f)$ be both a morphism $FX\to FY$ and $FY\to FX$. This can only happen if $FX=FY$. Then, given $f\cl X\to Y$ and $g\cl Y\to Z$, we have $FX=FY=FZ$ and we also require $F(g)\circ F(f)=F(f)\circ F(g)$. So $F$ sends every morphism in $\mathcal{C}$ to an endomorphism in $\mathcal{D}$, such that composites of morphisms in $\mathcal{C}$ are sent to composites of commuting endomorphisms in $\mathcal{D}$.

For instance, if $G$ is an abelian group and $\mathcal{D}$ is the category obtained from $G$ as in Exercise 1.3 (b), then every functor $F\cl \mathcal{C}\to \mathcal{D}$ is both covariant and contravariant.
\es

\bx
Show that there is a universal example for contravariant functors out of a category $\mathcal{C}$. That is, show that there is a category $\mathcal{C}^\op$ and a contravariant functor $\mathcal{C} \to \mathcal{C}^\op$ so that every other contravariant functor $\mathcal{C} \to \mathcal{D}$ has a unique factorization $\mathcal{C} \to \mathcal{C}^\op \to \mathcal{D}$, where the functor $\mathcal{C}^\op \to\mathcal{D}$ is covariant.
\ex

\bs
Define $\mathcal{C}^\op$ by setting $\ob(\mathcal{C}^\op)=\ob(\mathcal{C})$ and $\mor_{\mathcal{C}^\op}(X,Y)=\mor_{\mathcal{C}}(Y,X)$, for all $X,Y\in\ob(\mathcal{C}^\op)$. Define a functor $(-)^\op\cl \mathcal{C}\to \mathcal{C}^\op$ to be the identity on objects and by $(f\cl X\to Y)^\op=f^\op\cl Y\to X$ on morphisms. Note that $(\mathcal{C}^\op)^\op=\mathcal{C}$.

Let $F\cl \mathcal{C}\to \mathcal{D}$ be a contravariant functor and define $F'\cl\mathcal{C}^\op\to\mathcal{D}$ by $F'X=FX$ on objects and
\bse
F'(f^\op\cl Y\to X)=F(f\cl X\to Y)= F(f)\cl FY \to FX=F(f)\cl F'Y \to F'X
\ese
on morphisms, so that $F'$ is covariant. More succinctly, $F'=F\circ(-)^\op$. Then $F$ factors as $F=F'\circ(-)^\op$.

Now suppose that $G\cl \mathcal{C}^\op\to\mathcal{D}$ is another covariant functor such that $F=G\circ(-)^\op$. Then $F'X=FX=G(X^\op)=GX$ for all $X\in \mathcal{C}$ and, given any morphism $f\cl X\to Y$ in $\mathcal{C}$, we have
\bse
F'(f^\op)=F(f)=(G\circ(-)^\op)(f)=G(f^\op),
\ese
so $F'=G$.
\es

\bx
Consider the categories $\mathbf{AbGrp}$ of abelian groups (and homomorphisms) and $\mathbf{Sets}$ of sets (and functions). Since an abelian group is a set together with extra structure, we can define $F \cl \mathbf{AbGrp} \to\mathbf{Sets}$ by
\bse
G \longmapsto \fbox{$G$, but completely forgetting the group structure}.
\ese
Complete the definition of $F$ on morphisms, and show that $F$ is a functor.
\ex

\bs
The functor $F$ acts on morphisms as
\bse
f \longmapsto \fbox{$f$, but completely forgetting that it is a homomorphism}.
\ese
We obviously have $F(\id_G)=\id_{FG}$ since the identity homomorphism is, in particular, the identity function on the underlying set of $G$. Similarly, $F(g\circ f)=F(g)\circ F(f)$ since the composite homomorphism of $f\cl G\to H$ and $g\cl H\to K$ is, in particular, the composite of the functions between the underlying sets of $G$, $H$ and $K$. 
\es

\bx
Try to make a formal definition of `forgetful functor'.
\ex

\bs
While there is no formal definition of the term `forgetful functor', it is frequently used to refer to functors which `forget' some structure or property in an obvious way. For instance, there are forgetful functors from $\mathbf{Grp}$, $\mathbf{Top}$, $R$-$\mathbf{Mod}$, etc.\ to $\mathbf{Sets}$, as well as from $\mathbf{AbGrp}$ to $\mathbf{Grp}$, from $R$-$\mathbf{Mod}$-$S$ to $R$-$\mathbf{Mod}$, and so on.

It is often used in contrast with the term `free', so that if forgetful functors forget structure or properties, free functors add structure and properties in a canonical way, with minimal constraints. For example, we have a functor $F\cl \mathbf{Sets}\to\mathbf{Grp}$ which associates to each set $S$, the free group $F(S)$ generated by $S$; functors $I,D\cl \mathbf{Sets}\to\mathbf{Top}$ endowing sets with the trivial and discrete topology, respectively; the abelianisation functor $\mathbf{Grp}\to\mathbf{AbGrp}$ sending each group $G$ to $G/[G,G]$, the largest abelian quotient of $G$; etc.
\es

\bx
Let $\mathcal{V}$ denote the category $\mathbf{Vect}_{\R}$ of all vector spaces over the real numbers and all linear transformations. Thus
\bse
\mor_{\mathcal{V}}(V,W) = \Hom_{\R}(V,W) = \{T \cl V \to W \mid T \text{ is $\R$-linear}\}.
\ese
\ben[label=(\alph*)]
\item Define $F \cl \mathcal{V} \to\mathcal{V}$ by the rules
\bse
F(V) = \Hom_{\R}(\R, V) \qquad \text{and} \qquad F(f) \cl g 	\mapsto f \circ g.
\ese
Show that $F$ is a covariant functor.
\item Define $G \cl \mathcal{V} \to\mathcal{V}$ by the rules
\bse
G(V) = \Hom_{\R}(V,\R) \qquad \text{and} \qquad G(f) \cl g 	\mapsto g \circ f.
\ese
Show that $G$ is a contravariant functor.
\een
\ex

\bs
\ben[label=(\alph*)]
\item First note that $F(\id_V)\cl  \Hom_{\R}(\R, V)\to  \Hom_{\R}(\R, V)$ is given by $g\mapsto \id_V\circ g=g$, so $F(\id_V)=\id_{F(V)}$. Now let $f\cl U\to V$ and $g\cl V\to W$ be $\R$-linear maps in $\mathcal{V}$ and let $h\in \Hom_{\R}(\R, U)$. 
Then
\bse
[F(g)\circ F(f)] (h)=F(g)(f\circ h)=g\circ (f\circ h)
\ese
and
\bse
F(g\circ f) (h)=(g\circ f)\circ h,
\ese
which are equal by associativity. So $F(g\circ f) =F(g)\circ F(f)$.
\item As above, note that $G(\id_V)\cl \Hom_{\R}(V,\R)\to \Hom_{\R}(V,\R)$ is given by $g\mapsto g\circ \id_V=g$, so $G(\id_V)=\id_{G(V)}$. Now let $f\cl U\to V$ and $g\cl V\to W$ be $\R$-linear maps in $\mathcal{V}$ and let $\Hom_{\R}(W,\R)$. Then
\bse
[G(f)\circ g(g)] (h)=G(f)(h\circ g)=(h\circ g)\circ f
\ese
and
\bse
G(g\circ f) (h)=h\circ (g\circ f),
\ese
which are equal by associativity. So $G(g\circ f) =G(f)\circ G(g)$.
\een
\es

\addtocounter{exercise}{1}
\bp
Prove Proposition 1.21: Let $\mathcal{C}$ be a category, and let $A,B\in \ob(\mathcal{C})$. Prove the following statements.
\ben[label=(\alph*)]
\item For $f\cl X\to Y$, write $f^*\cl\mor_{\mathcal{C}}(Y,B)\to\mor_{\mathcal{C}}(X,B)$ for the function $f^*\cl g\mapsto g\circ f$. Then the rules
\bse
F(X)=\mor_{\mathcal{C}}(X,B) \qquad \text{and} \qquad F(f)=f^*\cl F(Y)\to F(X)
\ese
define a contravariant functor from $\mathcal{C}$ to $\mathbf{Sets}$.
\item For $f\cl X\to Y$, write $f_*\cl\mor_{\mathcal{C}}(A,X)\to\mor_{\mathcal{C}}(A,Y)$ for the function $f_*\cl g\mapsto f\circ g$. Then the rules
\bse
G(X)=\mor_{\mathcal{C}}(A,X) \qquad \text{and} \qquad G(f)=f_*\cl G(X)\to G(Y)
\ese
define a covariant functor from $\mathcal{C}$ to $\mathbf{Sets}$.
\een
{\scshape Hint}. Simply generalize your work from Exercise 1.20.
\ep

\bs
\ben[label=(\alph*)]
\item First note that $F(\id_X)\cl F(X)\to F(X)$ is given by $g\mapsto g\circ \id_X=g$, so $F(\id_X)=\id_{F(X)}$. Now let $f\cl X\to Y$ and $g\cl Y\to Z$ be morphisms in $\mathcal{C}$ and let $h\in F(Z)$. Then
\bse
[F(f)\circ F(g)] (h)=F(f)(h\circ g)=(h\circ g)\circ f
\ese
and
\bse
F(g\circ f) (h)=h\circ (g\circ f),
\ese
which are equal by associativity. So $F(g\circ f) =F(f)\circ F(g)$.
\item As above, note that $G(\id_X)\cl F(X)\to F(X)$ is given by $g\mapsto \id_X\circ g=g$, so $G(\id_X)=\id_{G(X)}$. Now let $f\cl X\to Y$ and $g\cl Y\to Z$ be morphisms in $\mathcal{C}$ and let $h\in G(X)$. Then
\bse
[G(g)\circ G(f)] (h)=G(g)(f\circ h)=g\circ (f\circ h)
\ese
and
\bse
G(g\circ f) (h)=(g\circ f)\circ h,
\ese
which are equal by associativity. So $G(g\circ f) =G(g)\circ G(f)$.
\een
\es


\bp
Let $f\cl A \to B$ and $g\cl X\to Y$ in a category $\mathcal{C}$. Show that
\bse
f^*\circ g_* = g_*\circ f^* \cl \mor_{\mathcal{C}}(B,X)\to \mor_{\mathcal{C}}(A,Y).
\ese
\ep

\bs
This is a consequence of associativity. Let $h\in \mor_{\mathcal{C}}(B,X)$. Then
\bi{rCl}
(f^*\circ g_*)(h)=(g\circ h)\circ f = g\circ (h\circ f)=(g_*\circ f^*)(h).
\ei
\es

\bp
Let $f\cl A \to B$ be a morphism in a category $\mathcal{C}$.
\ben[label=(\alph*)]
\item Suppose that the induced map $f_* \cl \mor_{\mathcal{C}}(X,A)\to \mor_{\mathcal{C}}(X,B)$ is a bijection for every $X$. Show that $f$ is an equivalence.
\item Suppose that the induced map $f^* \cl \mor_{\mathcal{C}}(B,X)\to \mor_{\mathcal{C}}(A,X)$ is a bijection for every $X$. Show that $f$ is an equivalence.
\een
{\scshape Hint}. Try plugging in $X=A$ and $X=B$.
\ep

\bs
\ben[label=(\alph*)]
\item Take $X=B$. As $f_*\cl \mor_{\mathcal{C}}(B,A)\to \mor_{\mathcal{C}}(B,B)$ is a bijection, we can define $g\cl B\to A$ by $g:=f_*^{-1}(\id_B)$. Then we immediately have
\bse
f\circ g = f_*(g)=f_*(f_*^{-1}(\id_B))=\id_B.
\ese
Furthermore, note that we have
\bi{rCl}
f_*(g\circ f)& = &f\circ g \circ f=\id_B\circ f =f\\
f_*(\id_A) & = & f\circ \id_A = f.
\ei
As $f_*\cl \mor_{\mathcal{C}}(A,A)\to \mor_{\mathcal{C}}(A,B)$ is a bijection, we must have $g\circ f=\id_A$. So $f$ is an equivalence.
\item Similarly, take $X=A$ and define $g\cl B\to A$ by $g:=(f^*)^{-1}(\id_A)$. Then we have $g\circ f=f^*(g)=\id_A$ and, as $f^*(f\circ g)=f=f^*(\id_B)$, we also have $f\circ g=\id_B$.
\een
\es


\section{Natural Transformations}

\bp
Let $\phi \cl A \to B$ be a morphism in $\mathcal{C}$.
\ben[label=(\alph*)]
\item Define two functors $\mathcal{C} \to \mathbf{Sets}$ by the rules $F(X) = \mor_{\mathcal{C}}(X,A)$ and $G(X) = \mor_{\mathcal{C}}(X,B)$. Show that $\Phi_X = \phi_* \cl F \to G$ is a natural transformation.
\item Define two functors $\mathcal{C} \to \mathbf{Sets}$ by the rules $H(X) = \mor_{\mathcal{C}}(A,X)$ and $I(X) = \mor_{\mathcal{C}}(B,X)$. Show that $\Phi_X = \phi^* \cl I \to H$ is a natural transformation.
\een
\ep

\bs
\ben[label=(\alph*)]
\item Let $f\cl X\to Y$ be a morphism in $\mathcal{C}$. We have to show that the diagram
\bse
\begin{tikzcd}[row sep=large,column sep=large]
F(Y) \ar[r,"F(f)"]\ar[d,"\Phi_Y"] & F(X)\ar[d,"\Phi_X"]\\
G(Y)\ar[r,"G(f)"]& G(X)
\end{tikzcd}
\ese
commutes. The morphisms $\Phi_X$ and $\Phi_Y$ are both $\phi_*$, i.e.\ post-composition with $\phi$, while $F(f)$ and $G(f)$ are both pre-composition with $f$, as in Problem 1.22. Thus the diagram reads
\bse
\begin{tikzcd}[row sep=large,column sep=large]
\mor_{\mathcal{C}}(Y,A) \ar[d,"\phi\circ -"]\ar[r,"-\circ f"] & \mor_{\mathcal{C}}(X,A)\ar[d,"\phi\circ -"] \\
\mor_{\mathcal{C}}(Y,B) \ar[r,"-\circ f"] & \mor_{\mathcal{C}}(X,B) 
\end{tikzcd}
\ese
Let $h\in\mor_{\mathcal{C}}(Y,A)$. Chasing $h$ around the diagram, we obtain $\phi\circ(h\circ f)$ going across and down, and $(\phi \circ h)\circ f$ going down and across, which are obviously equal by associativity. So $\Phi$ is a natural transformation.
\item Similarly to part (a), we let $f\cl X\to Y$ be a morphism in $\mathcal{C}$ and observe that the diagram
\bse
\begin{tikzcd}[row sep=large,column sep=large]
I(X) \ar[r,"I(f)"]\ar[d,"\Phi_X"] & I(Y)\ar[d,"\Phi_Y"]\\
H(X)\ar[r,"H(f)"]& H(Y)
\end{tikzcd}
\ese
reads
\bse
\begin{tikzcd}[row sep=large,column sep=large]
\mor_{\mathcal{C}}(B,X) \ar[d,"-\circ \phi"]\ar[r,"f\circ -"] & \mor_{\mathcal{C}}(B,Y)\ar[d,"-\circ \phi"] \\
\mor_{\mathcal{C}}(A,X) \ar[r,"f\circ -"] & \mor_{\mathcal{C}}(A,Y) 
\end{tikzcd}
\ese
and, again, commutes by associativity. 
\een
\es

\bx
This problem refers to the functors $F(V) = \Hom_{\R}(\R,V)$ and $G(V) = \Hom_{\R}(V,\R)$ of Exercise 1.20.
\ben[label=(\alph*)]
\item Show that for every vector space $V$, $V\cong F(V)$. Define a natural isomorphism $\Theta\cl F \to \id_{\mathcal{V}}$.
\item Show that for every finite-dimensional vector space $V$, $V\cong G(V)$. Show that there is no \textbf{natural} isomorphism $\Theta \cl G \to \id$, even if you restrict your attention just to finite-dimensional vector spaces.
\een
\ex

\bs
\ben[label=(\alph*)]
\item Define $\Phi_V\cl V\to F(V)=\Hom_{\R}(\R,V)$ by 
\bi{rrCl}
\Phi_V(v)\cl & \R & \to & V\\
& t & \mapsto &tv.
\ei
First, note that, for all $\lambda,t_1,t_2\in\R$, 
\bi{rCl}
\Phi_V(v)(\lambda t_1+t_2) & = & (\lambda t_1+t_2)v\\
& =&\lambda\lambda t_1v+t_2v\\
& =&\lambda\Phi_V(v)(t_1)+\Phi_V(v)(t_2),
\ei
so that we indeed have $\Phi_V(v)\in\Hom_{\R}(\R,V)$, for all $v\in V$. Moreover,
\bi{rCl}
\Phi_V(\lambda v_1+v_2)(t) & = & t(\lambda v_1+v_2)\\
& = & \lambda tv_1+tv_2\\
& = & \lambda\Phi_V( v_1)(t)+\Phi_V(v_2)(t)\\
& = & (\lambda\Phi_V( v_1)+\Phi_V(v_2))(t)
\ei
for all $t\in\R$, so $\Phi_V(\lambda v_1+v_2)=\lambda\Phi_V( v_1)+\Phi_V(v_2)$ for all $\lambda \in \R$ and $v_1,v_2\in V$. Thus $\Phi_V$ itself is a linear transformation $V\to F(V)$.

Now define $\Psi_V\cl \Hom_{\R}(\R,V)\to V$ by $\Psi_V(f)=f(1)$. We have
\bi{rCl}
\Psi_V(\lambda f_1+f_2) & = & (\lambda f_1+f_2)(1)\\
& = & \lambda f_1(1)+f_2(1)\\
& = & \lambda\Psi_V( f_1)+\Psi_V(f_2)
\ei
for all $\lambda \in \R$ and $f_1,f_2\in\Hom_{\R}(\R,V)$, so $\Psi_V$ is a linear transformation $F(V)\to V$. Now let $v\in V$ and $f\in \Hom_{\R}(\R,V)$. Observe that
\bse
\Psi_V(\Phi_V(v))=\Phi_V(v)(1)=1v=v
\ese
and
\bse
\Phi_V(\Psi_V(f))(t)=\Phi_V(f(1))(t)=tf(1)=f(t) 
\ese
for all $t\in \R$, so $\Phi_V(\Psi_V(f))=f$. Thus $\Phi_V$ and $\Psi_V$ are inverse linear isomorphisms, and hence $V\cong F(V)$ for all $V\in \mathcal{V}$.

Now let $f\cl V\to W$ be a linear transformation and consider the diagram
\bse
\begin{tikzcd}[row sep=large,column sep=large]
\Hom_{\R}(\R,V) \ar[r,"f\circ -"] \ar[d,"\Psi_V"]& \Hom_{\R}(\R,W)\ar[d,"\Psi_W"]\\
V \ar[r,"f"]& W
\end{tikzcd}
\ese
For any $h\in\Hom_{\R}(\R,V)$, we have 
\bse
f(\Psi_V(h))=f(h(1))=(f\circ h)(1)=\Psi_W(f\circ h),
\ese
so the diagram commutes. Hence, the linear transformations $\Psi_V$ assemble into a natural transformation $\Theta\cl F \to \id_{\mathcal{V}}$, with $\Theta_V=\Psi_V$. Since each $\Psi_V$ has an inverse $\Phi_V$, we conclude that $\Theta\cl F \to \id_{\mathcal{V}}$ is a natural isomorphism.
\item Suppose that $V$ is finite dimensional, say $\dim V=n$. Thus $V$ admits a basis $\{e_1,\ldots,e_n\}$. Define $f^1,\ldots,f^n\in\Hom_{\R}(V,\R)$ by requiring
\bse
f^i(e_j)=\delta^i_j=\begin{cases}1&\text{ if }i=j\\0&\text{ if }i\neq j\end{cases}
\ese
and extending linearly. Let $\lambda_1,\ldots,\lambda_n\in\R$ and suppose that
\bse
\sum_{i=1}^n\lambda_if^i=0.
\ese
Applying both sides to each $e_j$, for $1\leq j\leq n$, yields
\bse
0=\biggl(\sum_{i=1}^n\lambda_if^i\biggr)e_j=\sum_{i=1}^n\lambda_if^i(e_j)=\sum_{i=1}^n\lambda_i\delta^i_j=\lambda_j.
\ese
So $\lambda_j=0$ for all $1\leq j\leq n$ and hence $\{f^1,\ldots,f^n\}$ is linearly independent.

Now let $h\in \Hom_{\R}(V,\R)$ and let $v=\sum_{i=1}^nv^ie_i\in V$. Then, noting that
\bse
f^j(v)=f^j\biggl(\sum_{i=1}^nv^ie_i\biggr)=\sum_{i=1}^nv^if^j(e_i)=\sum_{i=1}^nv^i\delta^j_i=v^j,
\ese
we have
\bse
h(v)=h\biggl(\sum_{i=1}^nv^ie_i\biggr)=\sum_{i=1}^nv^ih(e_i)=\sum_{i=1}^nf^i(v)h(e_i)=\biggl(\sum_{i=1}^nh(e_i)f^i\biggr)v,
\ese
so $h=\sum_{i=1}^nh(e_i)f^i$ and hence $\{f^1,\ldots,f^n\}$ is a basis of $\Hom_{\R}(V,\R)$.

Thus $V\cong \R^n\cong \Hom_{\R}(V,\R)$, so $V\cong G(V)$ for all finite-dimensional $V$.

Now observe that $G$ is a contravariant functor $\mathcal{V}\to \mathcal{V}$ while $\id_{\mathcal{V}}$ is a covariant functor $\mathcal{V}\to \mathcal{V}$. Natural transformations are only defined between functors of the same variance, so there are no natural transformations, much less natural isomorphisms, from $G$ to $\id_{\mathcal{V}}$.

There is, however, a sensible notion of a morphism $F\to G$ when $F$ and $G$ are of opposite variance, called a \emph{dinatural transformation}. Given functors $F,G\cl \mathcal{C}\to\mathcal{D}$ such that $F$ is covariant and $G$ is contravariant, a dinatural transformation $\Phi\cl F\to G$ consists in a collection of morphisms
\bse
\{\Phi_X\cl F(X)\to G(X)\mid X\in \mathcal{C}\}
\ese
in $\mathcal{D}$ such that, for each morphism $f\cl X\to Y$ in $\mathcal{C}$, the diagram
\bse
\begin{tikzcd}[row sep=large,column sep=large]
F(X) \ar[d,"\Phi_X"]\ar[r,"F(f)"]& F(Y)\ar[d,"\Phi_Y"]\\
G(X) & \ar[l,"G(f)"]G(Y)
\end{tikzcd}
\ese
commutes in $\mathcal{D}$, i.e.\ $\Phi_X=G(f)\circ\Phi_Y\circ F(f)$. A dinatural isomorphism is a dinatural transformation $\Phi$ in which each component map $\Phi_X$ is an isomorphism in $\mathcal{D}$. One can obviously reverse the direction of $F(f)$ and $G(f)$ when $F$ is contravariant and $G$ covariant.

The statement that $V$ is not canonically isomorphic to $\Hom_{\R}(V,\R)$ amounts to saying that there is no dinatural isomorphism $G\to \id_{\mathcal{V}}$. Indeed, suppose that there is a dinatural isomorphism $\Phi\cl G\to \id_{\mathcal{V}}$. Then, the square
\bse
\begin{tikzcd}[row sep=large,column sep=large]
\Hom_{\R}(V,\R)  \ar[d,"\Phi_V"]& \ar[l,"-\circ f"']\Hom_{\R}(W,\R)\ar[d,"\Phi_W"]\\
V \ar[r,"f"]& W
\end{tikzcd}
\ese
commutes for any linear transformation $f\cl V\to W$. In particular, for the zero-transformation $0\cl V\to W$, we have
\bse
\Phi_W=0\circ\Phi_V\circ0=0,
\ese
so $\Phi_W$ cannot be an isomorphism (except when $W=0$). As this is true for all $W\in\mathcal{V}$, we conclude that the only dinatural transformation $G\to \id_{\mathcal{V}}$ is the zero-transformation, which is obviously not an isomorphism. Restricting to finite-dimensional vector spaces does not change the argument.
\een
\es

\addtocounter{exercise}{1}

\bp
Let $F,G, H, I \cl \mathcal{C} \to \mathbf{Sets}$ be the functors defined in Problem 1.25.
\ben[label=(\alph*)]
\item If $\Phi\cl F\to G$ is a natural transformation, show that there is a unique map $\phi \cl A \to B$ such that $\Phi_X = \phi_*$ for every $X \in \mathcal{C}$.
\item If $\Phi \cl I \to H$ is a natural transformation, show that there is a unique map $\phi\cl A \to B$ such that $\Phi_X = \phi^*$ for every $X \in \mathcal{C}$.
\een
Notice that your proof of (b) is formally very similar to your proof of (a). Can you be precise about how the two proofs are related?

{\scshape Hint}. In both cases, the domain of $\phi$ is $A$, and the target is $B$.
\ep

\bs
\ben[label=(\alph*)]
\item
\item
\een
\es

\bp
Let $A,B \in \mathcal{C}$, and use them to define functors $F(X) = \mor_{\mathcal{C}}(X,A)$ and $G(X) = \mor_{\mathcal{C}}(X,B)$.
\ben[label=(\alph*)]
\item Suppose there is a natural isomorphism $\Phi \cl F \to G$. Show that $A\cong B$.
\item Show that (a) is false without the word `natural' --- that is, make up an example where $F(X)\cong G(X)$ for all $X$, but where, nonetheless, $A\not\cong B$.

{\scshape Hint}. Your category must have at least two objects; can it have exactly two objects?
\item Prove that $A$ and $B$ are isomorphic if the functors $\mor_{\mathcal{C}}(A,-)$ and $\mor_{\mathcal{C}}(B,-)$ are naturally equivalent.
\een
\ep

\bs
\ben[label=(\alph*)]
\item
\item
\item
\een
\es

\bx
Show that an equivalence of categories need be neither injective nor surjective on objects.
\ex

\bs
\es

\section{Duality}

\bx
\ben[label=(\alph*)]
\item The notation $f \cl X \to Y$ is shorthand for the sentence: `$f$ is a morphism with domain $X$ and target $Y$.' What is the dual of this statement?
\item Find instances of duality in the previous sections.
\een
\ex

\bs
\ben[label=(\alph*)]
\item The dual of the given statement is `$f^\op$ is a morphism with domain $Y$ and target $X$', denoted $f^\op\cl Y\to X$.
\item Parts (a) and (b) of Exercise 1.20 and Problems 1.22, 1.24, 1.25 and 1.28 are all dual to one another.
\een
\es

\begin{extrae}
\label{extr:liftext}
Show how the problem: `decide whether $f \cl X \to Y$ is a homeomorphism' can be written in terms of extension and/or lifting problems.
\end{extrae}

\bs
The lifting problem for $h\cl Z\to Y$ and $f\cl X\to Y$ consists in finding a `lift' $\lambda\cl Z\to X$ such that $f\circ \lambda=h$. Let $Z=Y$ and $h=\id_Y$. Then the original lifting problem reduces to finding a map $\lambda\cl Y\to X$ such that the diagram
\bse
\begin{tikzcd}[row sep=large,column sep=large]
&& X\ar[d,"f"]\\
Y \ar[urr,bend left=10,"\lambda"] \ar[rr,"\id_Y"] & & Y
\end{tikzcd}
\ese
commutes, i.e.\ $f\circ \lambda=\id_Y$. The dual extension problem consists in finding a map $\epsilon\cl Y\to X$ such that the diagram
\bse
\begin{tikzcd}[row sep=large,column sep=large]
&& \ar[dll,bend right=10,"\epsilon"'] Y\\
X  & & \ar[ll,"\id_X"'] X\ar[u,"f"']
\end{tikzcd}
\ese
commutes, i.e.\ $\epsilon\circ f=\id_X$.

Thus, $f\cl X\to Y$ is a homeomorphism if, and only if, there exists a map $g\cl Y\to X$ such that $g=\lambda=\epsilon$ solves both of the above problems.
\es

\bp
Verify that the dual of each rule for a category is also a rule for a category, and likewise for functors and natural transformations.
\ep

\bs
\es

\section{Products and Sums}

\bp
Suppose $X,Y\in \mathcal{C}$ and the objects $P$ and $Q$ are both products for $X$ and $Y$. Show that $P\cong Q$.
\ep

\bs
Since $P$ and $Q$ are products of $X$ and $Y$, there are two pairs of morphisms $X\xleftarrow{\ p_X}P\xrightarrow{\ p_Y}Y$ and $X\xleftarrow{\ q_X}Q\xrightarrow{\ q_Y}Y$ satisfying the universal property of a product. Then, by the universal properties of $P$ and $Q$, there are (unique) morphisms $f\cl P\to Q$ and $g\cl Q\to P$ making the diagrams
\bse
\begin{tikzcd}[column sep=large,row sep=large]
& P \ar[dl,"p_X"']\ar[d,dashed,"f"]\ar[dr,"p_Y"]&\\
X & Q\ar[l,"q_X"] \ar[r,"q_Y"'] & Y
\end{tikzcd}
\qquad \quad
\begin{tikzcd}[column sep=large,row sep=large]
& Q \ar[dl,"q_X"']\ar[d,dashed,"g"]\ar[dr,"q_Y"]&\\
X & P\ar[l,"p_X"] \ar[r,"p_Y"'] & Y
\end{tikzcd}
\ese
commute. This implies that the diagrams
\bse
\begin{tikzcd}[column sep=large,row sep=large]
& P \ar[dl,"p_X"']\ar[d,"f"]\ar[dr,"p_Y"]&\\
X & Q\ar[l,"q_X"] \ar[r,"q_Y"'] \ar[d,"g"]& Y\\
& \ar[ul,"p_X"]P\ar[ur,"p_Y"'] &
\end{tikzcd}
\qquad \quad
\begin{tikzcd}[column sep=large,row sep=large]
& Q \ar[dl,"q_X"']\ar[d,"g"]\ar[dr,"q_Y"]&\\
X & P\ar[l,"p_X"] \ar[r,"p_Y"'] \ar[d,"f"]& Y\\
& \ar[ul,"q_X"]Q\ar[ur,"q_Y"'] &
\end{tikzcd}
\ese
also commute. In particular, we have
\bi{rClcrCl}
p_X\circ(g\circ f)&=&p_X &\qquad & q_X\circ(f\circ g)&=&q_X \\
p_Y\circ(g\circ f)&=&p_Y && q_Y\circ(f\circ g)&=&q_Y.
\ei
But by the universal properties of $P$ and $Q$, the unique morphisms $P\to P$ and $Q\to Q$ satisfying the above equations are $\id_P$ and $\id_Q$, respectively. Hence, $f$ and $g$ are equivalences.
\es

\bx
Let $X,Y\in\mathcal{C}$, and suppose $X\times Y$ exists. Explicitly define a bijection
\bse
\mor_{\mathcal{C}}(Z,X\times Y) \xrightarrow{\ \cong\ }\mor_{\mathcal{C}}(Z,X) \times \mor_{\mathcal{C}}(Z, Y) .
\ese
\ex

\bs
Let $\pr_X$ and $\pr_Y$ be the projections. Define
\bi{rrCl}
\phi\cl & \mor_{\mathcal{C}}(Z,X\times Y) & \to & \mor_{\mathcal{C}}(Z,X) \times \mor_{\mathcal{C}}(Z, Y) \\
& f & \mapsto & (\pr_X\circ f,\pr_Y\circ f).
\ei
By the universal property of the product, for each $(f,g)\in \mor_{\mathcal{C}}(Z,X) \times \mor_{\mathcal{C}}(Z, Y)$, there is a unique $h\in\mor_{\mathcal{C}}(Z,X\times Y)$ such that $f=\pr_X\circ h$ and $g=\pr_Y\circ h$, i.e.\ such that $\phi(h)=(f,g)$. Existence amounts to surjectivity, and uniqueness implies injectivity. Then $\phi(h)=(f,g)$ if, and only if, $\pr_X\circ h=f$ and $\pr_Y\circ h=g$. 

In light of this bijection, we will write $(f,g)$ for the map $h=\phi^{-1}(f,g)$. With this notation, we have $\pr_X\circ(f,g)=f$ and $\pr_Y\circ(f,g)=g$.

Note that, given a map $k\cl W\to Z$, we have
\bse
\pr_X\circ(f,g)\circ k=f\circ k \qquad \text{ and } \qquad \pr_Y\circ(f,g)\circ k=g\circ k,
\ese
so the map $(f,g)\circ k\cl W\to X\times Y$ corresponds to the pair $(f\circ k,g\circ k)$, and hence, since the correspondence is bijective,
\bse
(f,g)\circ k=(f\circ k,g\circ k).
\ese
However, there is no corresponding result for post-composition since, in the hypothetical expression $l\circ (f,g)=(l\circ f,l\circ g)$, the map $l$ should be a map $X\times Y \to V$, $X\to V$ and $Y\to V$ all at once.
\es

\bx
\ben[label=(\alph*)]
\item Show that if one of the products $X\times(Y\times Z)$ and $(X\times Y)\times Z$ exists in $\mathcal{C}$, then so does the other, and they are isomorphic.
\item Let $f\cl A\to X$ and $g\cl B\to Y$, and suppose that the products $A\times B$ and $X\times Y$ can be formed in $\mathcal{C}$. Give an explicit definition for the product map
\bse
f\times g \cl A\times B \to X\times Y.
\ese
\een
\ex

\bs
{\scshape Note}: Even if this is not stated explicitly in the exercise, one should assume that both of the binary products $X\times Y$ and $Y\times Z$ exist. 
\ben[label=(\alph*)]
\item Suppose that $X\times(Y\times Z)$ exists. Then there are pairs of morphisms
\bi{c}
Y \xleftarrow{\ \pr_Y\ } Y\times Z \xrightarrow{\ \pr_Z\ } Z\\
X \xleftarrow{\ \pr_X\ } X\times (Y\times Z) \xrightarrow{\ \pr_{Y\times Z}\ } Y\times Z\\
X \xleftarrow{\ \pr_X'\ } X\times Y \xrightarrow{\ \pr_Y'\ } Y
\ei
satisfying the respective universal properties. We seek to show that the object $X\times(Y\times Z)$ together with projection morphisms
\bi{rCl}
(\pr_X,\pr_Y\circ\pr_{Y\times Z})\cl && X\times(Y\times Z) \longrightarrow X\times Y\\ 
\pr_Z\circ\pr_{Y\times Z} \cl &&  X\times(Y\times Z) \longrightarrow Z
\ei
is a product of $X\times Y$ and $Z$. Then it will follow that $(X\times Y)\times Z$ exists and, by Exercise 1.33, $X\times (Y\times Z)\cong (X\times Y)\times Z$.

Let $f\cl A\to X\times Y$ and $g\cl A\to Z$ be two morphisms. Then we have morphisms
\bi{rCl}
\pr_X'\circ f \cl && A \longrightarrow X\\
(\pr_Y'\circ f,g)\cl && A \longrightarrow Y\times Z
\ei
and hence there is a unique morphism $h\cl A \to X\times (Y\times Z)$ making the diagram
\bse
\begin{tikzcd}[column sep=large,row sep=large]
& A \ar[dl,"\pr_X'\circ f"']\ar[d,dashed,"h"]\ar[dr,"{(\pr_Y'\circ f,g)}"]&\\
X & X\times (Y\times Z) \ar[l,"\pr_X"] \ar[r,"\pr_{Y\times Z}"'] & Y\times Z
\end{tikzcd}
\ese
commute. Observe that
\bi{rCl}
(\pr_X,\pr_Y\circ\pr_{Y\times Z})\circ h & = & (\pr_X\circ h,\pr_Y\circ\pr_{Y\times Z}\circ h)\\
& = & (\pr_X'\circ f,\pr_Y\circ(\pr_Y'\circ f,g))\\
& = & (\pr_X'\circ f,\pr_Y'\circ f)\\
& = & f
\ei
and 
\bi{rCl}
(\pr_Z\circ\pr_{Y\times Z})\circ h & = & \pr_Z\circ(\pr_{Y\times Z}\circ h)\\
& = & \pr_Z\circ(\pr_Y'\circ f,g)\\
& = & g.
\ei
That is, $h$ makes the diagram
\bse
\begin{tikzcd}[column sep=large,row sep=huge]
&& A \ar[dll,"f"'] \ar[d,"h"] \ar[drr,"g"] &&\\
X\times Y && X\times (Y\times Z) \ar[ll,"{(\pr_X,\pr_Y\circ\pr_{Y\times Z})}"] \ar[rr,"\pr_Z\circ\pr_{Y\times Z}"'] && Z
\end{tikzcd}
\ese
commute. We need to show that it is the unique such morphism. Let $\widetilde h\cl A \to X\times (Y\times Z)$ be another morphism making the above diagram commute, i.e.\
\bi{rCl}
(\pr_X,\pr_Y\circ\pr_{Y\times Z})\circ\widetilde h&=&f\\
(\pr_Z\circ\pr_{Y\times Z})\circ\widetilde h&=&g.
\ei
Then, as $f=(\pr_X'\circ f,\pr_Y'\circ f)$, we have $\pr_X\circ \widetilde h=\pr_X'\circ f$. Furthermore, $\pr_Y\circ\pr_{Y\times Z}\circ \widetilde h=\pr_Y'\circ f$ together with the second equation above, imply that $\pr_{Y\times Z}\circ \widetilde h=(\pr_Y'\circ f,g)$. Hence $\widetilde h$ also makes the earlier diagram commute. But $h$ is the unique morphism making that diagram commute and thus $\widetilde h=h$.

An entirely analogous argument shows that if $(X\times Y)\times Z$ exists, then so does $X\times (Y\times Z)$, and $(X\times Y)\times Z\cong X\times (Y\times Z)$.

\item Given $f\cl A\to X$ and $g\cl B\to Y$, we can form the morphisms
\bse
f\circ\pr_A\cl A\times B\to X\qquad \text{and}\qquad g\circ \pr_B\cl A\times B\to Y.
\ese
Then we can define $f\times g=(f\circ\pr_A,g\circ\pr_B)$. Explicitly, $f\times g$ is the unique morphism making the following diagram commute.
\bse
\begin{tikzcd}[column sep=large,row sep=large]
A \ar[d,"f"] & \ar[l,"\pr_A"'] A\times B \ar[d,dashed,"f\times g"] \ar[r,"\pr_B"]& B\ar[d,"g"]\\
X & \ar[l,"\pr_X"]X\times Y \ar[r,"\pr_Y"']& Y
\end{tikzcd}
\ese
Note that this construction behaves well with respect to identities, in that $\id_A\times\id_A=\id_{A\times A}$, and with respect to composition. Indeed, given $h\cl X\to Z$ and $k\cl Y\to W$, the commutative diagram
\bse
\begin{tikzcd}[column sep=large,row sep=large]
A \ar[dd,bend right=60,"h\circ f"']\ar[d,"f"] & \ar[l,"\pr_A"'] A\times B \ar[d,dashed,"f\times g"] \ar[r,"\pr_B"]& B\ar[d,"g"]\ar[dd,bend left=60,"k\circ g"]\\
X \ar[d,"h"]& \ar[l,"\pr_X"]X\times Y \ar[d,dashed,"h\times k"] \ar[r,"\pr_Y"']& \ar[d,"k"]Y\\
Z & \ar[l,"\pr_Z"]Z\times W \ar[r,"\pr_W"']& W
\end{tikzcd}
\ese
shows that $(h\times k)\circ(f\times g)=(h\circ f)\times(k\circ g)$.
\een
\es

\bx
Formulate precise definitions of the product $\mathcal{C}\times \mathcal{D}$ of two categories and of a functor of two variables.
\ex

\bs
We set $\ob(\mathcal{C}\times \mathcal{D})=\ob(\mathcal{C})\times\ob(\mathcal{D})$ and, for any $(A,B),(A',B')\in\mathcal{C}\times \mathcal{D}$,
\bse
\mor_{\mathcal{C}\times \mathcal{D}}((A,B),(A',B'))=\mor_{\mathcal{C}}(A,B)\times \mor_{\mathcal{D}}(A',B').
\ese
That is, the objects and morphisms of $\mathcal{C}\times \mathcal{D}$ are pairs of objects and morphisms from $\mathcal{C}$ and $\mathcal{D}$. We then have $\id_{(A,B)}=(\id_A,\id_B)$ and composition is defined componentwise.

In fact, for small categories $\mathcal{C}$ and $\mathcal{D}$ (i.e.\ such that $\ob(\mathcal{C})$ and $\ob(\mathcal{D})$ are sets), the product $\mathcal{C}\times \mathcal{D}$ thus defined is the categorical product (see Section 1.6) in $\mathbf{Cat}$, the category of small categories and functors.

A functor $F\cl\mathcal{C}\times \mathcal{D}\to\mathcal{E}$ is called a functor of two variables, or bifunctor. A bifunctor is covariant (resp. contravariant) if, and only if, for all $A\in\mathcal{C}$ and $B\in\mathcal{D}$, the functors $F(A,-)\cl \mathcal{D}\to \mathcal{E}$ and $F(-,B)\cl \mathcal{C}\to \mathcal{E}$ are both covariant (resp. contravariant). For example, the functor $-\times -\cl \mathcal{C}\times \mathcal{C}\to\mathcal{C}$ is covariant.

Additionally, we have the notion of a `mixed' bifunctor, which is a functor $F\cl\mathcal{C}\times \mathcal{D}\to\mathcal{E}$ such that $F(A,-)\cl \mathcal{D}\to \mathcal{E}$ is covariant for all $A\in\mathcal{C}$, and $F(-,B)\cl \mathcal{C}\to \mathcal{E}$ is contravariant for all $B\in\mathcal{D}$, or vice versa. Note that the functor $F(A,-)$ acts by $f\mapsto F(\id_A,f)$ on morphisms, and similarly for $F(-,B)$. An example of a functor of this kind is the hom-functor on $\mathcal{C}$, $\mor_{\mathcal{C}}(-,-) \cl \mathcal{C} \times \mathcal{C} \to \mathcal{C}$, which is contravariant in the first variable and covariant in the second.
\es

\bp
Let $X$ and $Y$ be objects in a category $\mathcal{C}$.
\ben[label=(\alph*)]
\item Show that, for any map $f\cl X \to Y$, the diagram
\bse
\begin{tikzcd}[column sep=large,row sep=large]
X\ar[d,"f"] \ar[r,"\Delta"]& X\times X\ar[d,"f\times f"]\\
Y\ar[r,"\Delta"] & Y\times Y
\end{tikzcd}
\ese
commutes.
\item Show that the diagram
\bse
\begin{tikzcd}[column sep=large,row sep=large]
X\times Y \ar[dr,bend right=10,"\id"']\ar[r,"\Delta"] & \ar[d,"{(\pr_1,\pr_2)}"](X\times Y)\times(X\times Y)\\
& X\times Y
\end{tikzcd}
\ese
is commutative.
\een
\ep

\bs
\ben[label=(\alph*)]
\item First, observe that there is a unique morphism $h\cl X\to Y\times Y$ making the diagram
\bse
\begin{tikzcd}[column sep=large,row sep=large]
&X\ar[dl,"f"']\ar[d,dashed,"h"]\ar[dr,"f"]&\\
Y & Y\times Y \ar[l,"\pr_Y^1"]\ar[r,"\pr_Y^2"']& Y
\end{tikzcd}
\ese
commute. Now consider the following diagram.
\bse
\begin{tikzcd}[column sep=large,row sep=large]
X\ar[rr,bend left=40,"\id_X"]\ar[d,"f"] \ar[r,"\Delta"]& X\times X\ar[d,"f\times f"]\ar[r,"\pr_X^1"]& X\ar[d,"f"]\\
Y\ar[r,"\Delta"] \ar[rr,bend right=40,"\id_Y"] & Y\times Y\ar[r,"\pr_Y^1"]&Y
\end{tikzcd}
\ese
By definition of the diagonal map, the top and bottom sections commute and, by definition of $f\times f$, so does the right square. We thus have
\bse
\pr_Y^1\circ (f\times f)\circ \Delta  =  f\circ \pr_X^1\circ \Delta=f\circ \id_X=f
\ese
and
\bse
\pr_Y^1\circ\Delta\circ f  =  \id_Y\circ f = f.
\ese
By replacing $\pr_X^1$ and $\pr_Y^1$ with $\pr_X^2$ and $\pr_Y^2$, respectively, in the above diagram, we also obtain $\pr_Y^2\circ (f\times f)\circ \Delta =f$ and $\pr_Y^2\circ\Delta\circ f = f$. So both $(f\times f)\circ \Delta$ and $\Delta\circ f$ make the first diagram commute, and thus
\bse
(f\times f)\circ \Delta = h = \Delta\circ f.
\ese
\item First note that $(\pr_1,\pr_2)$ is the preimage under the bijection $\phi$ of Exercise 1.34 of the pair of composites
\bi{C}
\pr_1 = (X\times Y)\times(X\times Y) \xrightarrow{\ \pr^1_{X\times Y} \ } X\times Y \xrightarrow{\ \pr_X \ }X\\
\pr_2 = (X\times Y)\times(X\times Y) \xrightarrow{\ \pr^2_{X\times Y} \ } X\times Y \xrightarrow{\ \pr_Y \ }Y.
 \ei
By the universal property of products, there is a unique map $h$ making the following diagram commute
\bse
\begin{tikzcd}[row sep=large,column sep=large]
&X\times Y\ar[dl,"\pr_X"']\ar[d,dashed,"h"]\ar[dr,"\pr_Y"]&\\
X & \ar[l,"\pr_X"]X\times Y \ar[r,"\pr_Y"']& Y
\end{tikzcd}
\ese
Observe that we have
\bi{rCl}
\pr_X\circ(\pr_1,\pr_2)\circ\Delta & = & \pr_1\circ\Delta\\
& = & (\pr_X\circ\pr^1_{X\times Y})\circ \Delta\\
& = & \pr_X\circ\pr^1_{X\times Y}\circ(\id,\id) \\
& = & \pr_X\circ \id\\
& = & \pr_X
\ei
and, similarly, $\pr_Y\circ(\pr_1,\pr_2)\circ\Delta=\pr_Y$. Therefore, as both $(\pr_1,\pr_2)\circ\Delta$ and, trivially, $\id$ make the above diagram commute, we must have
\bse
(\pr_1,\pr_2)\circ\Delta = h = \id.
\ese
\een
\es

\bx
Explain how to view the diagonal map as a natural transformation.
\ex

\bs
Consider the assignment $\Sq\cl \mathcal{C}\to \mathcal{C}$ given by $\Sq(X)=X\times X$ on objects and $\Sq(f)=f\times f$ on morphisms. Then, by Exercise 1.35 (b), we have
\bse
\Sq(\id_X)=\id_X\times \id_X=\id_{X\times X}=\id_{\Sq(X)}
\ese
and
\bse
\Sq(g\circ f)=(g\circ f)\times (g\circ f)=(g\times g)\circ (f\times f)=\Sq(g)\circ \Sq(f),
\ese
so $\Sq$ is a covariant functor $\mathcal{C}\to\mathcal{C}$. The diagonal maps $\Delta_X\cl X\to X\times X$ can be assembled into a natural transformation $\Delta\cl \id_{\mathcal{C}}\to \Sq$. The naturality square
\bse
\begin{tikzcd}[row sep=large,column sep=large]
X \ar[r,"\Delta_X"]\ar[d,"f"] & \Sq(X) \ar[d,"{\Sq(f)}"]\\
Y \ar[r,"\Delta_Y"] & \Sq(Y)
\end{tikzcd}
\ese
is the same square as in Problem 1.37 (a).
\es

\bx
\ben[label=(\alph*)]
\item Show that the product of two sets $X$ and $Y$ is simply the ordinary cartesian product
\bse
X \times Y = \{(x, y) \mid x \in X, y \in Y \}.
\ese
\item What is the product of two abelian groups $G$ and $H$?
\een
\ex

\bs
\ben[label=(\alph*)]
\item Let $\pr_X$ and $\pr_Y$ be the standard projection maps, i.e.\ $(x,y)\mapsto x$ and $(x,y)\mapsto y$, respectively. Let $f\cl Z\to X$ and $g\cl Z\to Y$ be arbitrary maps and define $h\cl Z\to X\times Y$ by $h(z)=(f(z),g(z))$. Then $\pr_X\circ h=f$ and $\pr_Y\circ h=g$, so $h$ makes the diagram
\bse
\begin{tikzcd}[row sep=large,column sep=large]
&Z\ar[dl,"f"']\ar[d,"h"]\ar[dr,"g"]&\\
X & \ar[l,"\pr_X"]X\times Y \ar[r,"\pr_Y"']& Y
\end{tikzcd}
\ese
commute. To see that it is the unique such, suppose that $h'\cl z\to X\times Y$ also makes the diagram commute. Let $z\in Z$ and let $h'(z)=(x,y)\in X\times Y$. Then
\bse
x=\pr_X(h'(z))=(\pr_X\circ h')(z)=f(z)
\ese
and, similarly, $y=g(z)$. Therefore $h'=h$.
\item
\een
\es

\bp
\ben[label=(\alph*)]
\item Formulate the (dual) definition of a \textbf{coproduct}, which is denoted $X\sqcup Y$. Coproducts are also known as (categorical) \textbf{sums}.
\item Prove that if $X, Y \in \ob(\mathcal{C})$, then
\bse
\mor_{\mathcal{C}}(X \sqcup Y,B) \cong \mor_{\mathcal{C}}(X,B) \times \mor_{\mathcal{C}}(Y,B).
\ese
Write down the isomorphism explicitly. Thus we can (and will) describe maps $F \cl X\sqcup Y \to B$ with the notation $(f, g)$, where $f \cl X \to B$ and $g \cl Y \to B$.

{\scshape Hint}. This is formally dual to Exercise 1.34, so you should be able to prove this by simply inverting all the arrows in your previous proof.
\item Write down the definition of $f \sqcup g \cl A \sqcup B \to X \sqcup Y$.
\item Explain how to view $\sqcup$ as a functor.
\item The dual of the diagonal map is called the \textbf{folding map}, and we will denote it by the symbol $\nabla$. Define it explicitly in category-theoretical language, and explain how to view it as a natural transformation.
\een
\ep

\bs
\ben[label=(\alph*)]
\item Given $X,Y\in\mathcal{C}$, their coproduct or sum is an object of $\mathcal{C}$, denoted $X \sqcup Y$, together with two morphisms $\inj_X\cl X\to X\sqcup Y$ and $\inj_Y\cl Y\to X\sqcup Y$ such that, given any other two morphisms $f\cl X\to Z$ and $g\cl Y\to Z$, there is a unique morphism $t\cl X\sqcup Y\to Z$ making the diagram
\bse
\begin{tikzcd}[row sep=large,column sep=large]
X \ar[dr,"f"']\ar[r,"\inj_X"]& X\sqcup Y \ar[d,dashed,"t"]&\ar[l,"\inj_Y"'] Y\ar[dl,"g"]\\
& Z &
\end{tikzcd}
\ese
commute, i.e.\ such that $t\circ \inj_X=f$ and $t\circ \inj_Y=g$. 

\item Define
\bi{rrCl}
\psi\cl & \mor_{\mathcal{C}}(X\sqcup Y,B) & \to & \mor_{\mathcal{C}}(X,B) \times \mor_{\mathcal{C}}(Y, B) \\
& f & \mapsto & ( f\circ \inj_X,f\circ \inj_Y).
\ei
By the universal property of the coproduct, for each $(f,g)\in \mor_{\mathcal{C}}(X,B) \times \mor_{\mathcal{C}}(Y, B)$, there is a unique $h\in\mor_{\mathcal{C}}(X\sqcup Y,B)$ such that $f= h\circ \inj_X$ and $g=h\circ \inj_Y$, i.e.\ such that $\psi(h)=(f,g)$. Existence amounts to surjectivity, and uniqueness implies injectivity. Then $\psi(h)=(f,g)$ if, and only if, $h\circ\inj_X=f$ and $h\circ\inj_Y=g$. 

Using the notation $(f,g)$ for the map $h=\psi^{-1}(f,g)$, we have $(f,g)\circ\inj_X=f$ and $(f,g)\circ\inj_Y=g$.

Note that, given a map $k\cl B\to C$, we have
\bse
k\circ(f,g)\circ \inj_X=k\circ f \qquad \text{ and } \qquad k\circ(f,g)\circ \inj_Y=k\circ g,
\ese
so the map $k\circ (f,g)\cl  X\sqcup Y\to C$ corresponds to the pair $(k\circ f,k\circ g)$, and hence, since the correspondence is bijective,
\bse
k\circ(f,g)=(k\circ f,k\circ g).
\ese
However, there is no corresponding result for pre-composition since, in the hypothetical expression $(f,g)\circ l=(f\circ l,g\circ l)$, the map $l$ should be a map $D\to X\sqcup Y$, $D\to X$ and $D\to Y$ all at once.
\item Given $f\cl A\to X$ and $g\cl B\to Y$, we can form the morphisms
\bse
\inj_X\circ f\cl A\to X\sqcup Y\qquad \text{and}\qquad \inj_Y\circ g\cl B\to X\sqcup Y.
\ese
Then we can define $f\sqcup g:=(\inj_X\circ f,\inj_Y\circ g)$. Explicitly, $f\sqcup g$ is the unique morphism making the following diagram commute.
\bse
\begin{tikzcd}[column sep=large,row sep=large]
A  \ar[r,"\inj_A"]\ar[d,"f"] &  A\sqcup B \ar[d,dashed,"f\sqcup g"] & \ar[l,"\inj_B"'] B\ar[d,"g"]\\
X \ar[r,"\inj_X"] & X\sqcup Y & Y\ar[l,"\inj_Y"']
\end{tikzcd}
\ese
\item Note that the construction in part (c) behaves well with respect to identities, in that $\id_A\sqcup\id_A=\id_{A\sqcup A}$, and with respect to composition. Indeed, given $h\cl X\to Z$ and $k\cl Y\to W$, the commutative diagram
\bse
\begin{tikzcd}[column sep=large,row sep=large]
A \ar[dd,bend right=60,"h\circ f"']\ar[d,"f"] \ar[r,"\inj_A"] & A\sqcup B \ar[d,dashed,"f\sqcup g"] & \ar[l,"\inj_B"'] B\ar[d,"g"]\ar[dd,bend left=60,"k\circ g"]\\
X \ar[r,"\inj_X"'] \ar[d,"h"]& X\sqcup Y \ar[d,dashed,"h\sqcup k"] & \ar[l,"\inj_Y"]\ar[d,"k"]Y\\
Z \ar[r,"\inj_Z"'] & Z\sqcup W&  \ar[l,"\inj_W"]W
\end{tikzcd}
\ese
shows that $(h\sqcup k)\circ(f\sqcup g)=(h\circ f)\sqcup(k\circ g)$. We thus have a covariant functor of two variables $\sqcup\cl \mathcal{C}\times \mathcal{C}\to \mathcal{C}$. 
\item Similarly to diagonal map, the folding map $\nabla\cl X\sqcup X\to X$ is defined as $\nabla=(\id_X,\id_X)$, under the bijection in part (b). By definition, then, $\nabla\circ\inj_X=\id_X$.
\een
\es


\bx
\ben[label=(\alph*)]
\item Show that the sum of two sets $X$ and $Y$ is simply the disjoint union of $X$ and $Y$. Conclude that $X\sqcup Y$ and $X\times Y$ are not generally equivalent.
\item What is the folding map in the case $X = \{a, b, c\}$?
\item What is the sum of abelian groups $G$ and $H$? Construct a nice map $w \cl G \sqcup H \to G \times H$; what can you say about it?
\een
\ex

\bs
\ben[label=(\alph*)]
\item
\item
\item
\een
\es

\bx
\ben[label=(\alph*)]
\item Give number-theoretical interpretations of products and sums in the category of positive integers $1,2,3,\ldots$ with morphisms corresponding to divisibility.
\item Repeat (a) with the category of real numbers with morphisms corresponding to inequalities $x \leq y$.
\item Is there a category structure on the set $\N$ so that the categorical product is the same as the numerical product?
\een
\ex

\bs
\ben[label=(\alph*)]
\item
\item
\item
\een
\es


\section{Initial and Terminal Objects}

\bx
Find initial and terminal objects in the following contexts.
\ben[label=(\alph*)]
\item The category $\mathbf{Sets}$ of sets and functions.
\item The category $\mathbf{Top}$ of topological spaces and continuous functions.
\item The category $\mathbf{Grp}$ of groups and group homomorphisms.
\een
\ex

\bs
\ben[label=(\alph*)]
\item Given any set $X$, there is a unique map $\vn \to X$. 

Indeed, if a function $f\cl A\to B$ is defined to be a subset $f\subseteq A\times B$ such that for all $a\in A$ there is a unique $b\in B$ such that $(a,b)\in f$, then clearly $\vn \subseteq \vn \times X=\vn$ satisfies this condition vacuously for any $X$ (including the case $\vn\to\vn$) and it is the unique subset of $\vn$.

So $\vn$ is an initial object of $\mathbf{Sets}$ and, in fact, the unique such.
Note, however, that there is no function $Y\to \vn$ for non-empty $Y$ since the above existence condition fails. Hence $\vn$ cannot be a terminal object.

Instead, a terminal object of $\mathbf{Sets}$ is any singleton $\{x\}$ as, given any other set $Y$, there is a unique map $Y\to \{x\}$, namely, the constant map with value $x$. Of course, there are infinitely many terminal objects in $\mathbf{Sets}$, but since there is a unique map between any two singletons which is, obviously, a bijection, the terminal object of $\mathbf{Sets}$ is unique up to unique isomorphism.
\item The empty space $\vn$ with its unique topology $\{\vn\}$ is such that the unique map $\vn\to X$ to any topological space $X$ is continuous. So $(\vn,\{\vn\})$ is the (unique) initial object in $\mathbf{Top}$.

Similarly, any one-point space $P$ with its unique topology $\{\vn,P\}$ is such that the unique map $Y\to P$ for any topological space $Y$ is continuous. So $(P,\{\vn,P\})$ is a terminal object of $\mathbf{Top}$.

\item The trivial group $\{e\}$ in $\mathbf{Grp}$, which is unique up to unique group isomorphism, is a zero object of $\mathbf{Grp}$, i.e.\ it is both initial and terminal.    
Indeed, given any other group $G$, there is a unique homomorphism $G\to \{e\}$ (the unique set function $G\to \{e\}$) and a unique homomorphism $\{e\} \to G$, i.e.\ the function mapping $e$ to the identity element of $G$.
\een
\es

\bx
Verify that $\mathbf{Sets_*}$ is a pointed category. What is a sum in $\mathbf{Sets_*}$? What is a product?
\ex

\bs
Any pointed singleton $(\{x\},x)$ is a zero object of $\mathbf{Sets_*}$. Indeed, given any other pointed set $(Y,y)$, there is a unique pointed map $(Y,y)\to(\{x\},x)$ (the unique map $Y\to \{x\}$) and a unique pointed map $(\{x\},x)\to (Y,y)$ sending $x\mapsto y$.

The sum of $(X,x_0)$ and $(Y,y_0)$ in $\mathbf{Sets _*}$ is $(X\sqcup Y/\{x_0,y_0\},[x_0]=[y_0])$ with the usual injections, while their product is $(X\times Y,(x_0,y_0))$ with the usual projections. Indeed, given any pointed maps $f\cl(X,x_0)\to(Z,z_0)$ and $g\cl(Y,y_0)\to(Z,z_0)$, there is a unique map $h\cl X\sqcup Y\to Z$ such that $h\circ\inj_X=f$ and $h\circ\inj_Y=g$. Then, in particular, 
\bse
h(x_0)=h(\inj_X(x_0))=f(x_0)=z_0=g(y_0)=h(\inj_Y(y_0))=h(y_0).
\ese
So $h$ is, in fact, a well-defined pointed map $(X\sqcup Y/\{x_0,y_0\},[x_0])\to(Z,z_0)$, and the unique such making the diagram
\bse
\begin{tikzcd}[row sep=large,column sep=large]
(X,x_0) \ar[dr,"f"']\ar[r,"\inj_X"] & \bigl(\frac{X\sqcup Y}{\{x_0,y_0\}},[x_0]\bigr) \ar[d,dashed,"h"] & \ar[l,"\inj_Y"'](Y,y_0)\ar[dl,"g"]\\
&(Z,z_0)&
\end{tikzcd}
\ese
commute. Dually, given any pointed maps $f\cl(Z,z_0)\to(X,x_0)$ and $g\cl(Z,z_0)\to(Y,y_0)$, there is a unique map $h\cl Z\to X\times Y$ such that $\pr_X\circ h=f$ and $\pr_Y\circ h=g$. Then, in particular, 
\bse
\pr_X(h(z_0))=f(z_0)=x_0 \qquad \text{ and } \qquad \pr_Y(h(z_0))=g(z_0)=y_0,
\ese
so $h(z_0)=(x_0,y_0)$. Thus $h$ is indeed the unique pointed map making the following diagram commute.
\bse
\begin{tikzcd}[row sep=large,column sep=large]
&(Z,z_0)\ar[dl,"f"']\ar[d,dashed,"h"] \ar[dr,"g"] &\\
(X,x_0) & (X\times Y,(x_0,y_0)) \ar[l,"\pr_X"]\ar[r,"\pr_Y"'] & (Y,y_0)
\end{tikzcd}
\ese
\es

\bp
\ben[label=(\alph*)]
\item Suppose $\mathcal{C}$ is a category with a terminal object $\tau$, and let $X,Y\in\mathcal{C}$, and suppose that a product $P$ for $X$ and $Y$ exists in $\mathcal{C}$. Show that $P$ solves the problem expressed in the diagram
\bse
\begin{tikzcd}%[column sep=large,row sep=large]
Z\ar[dr,dashed,"\exists!\,t"]\ar[rrrd,bend left=30,"f"] \ar[dddr,bend right=30,"g"'] &&\\
&P \ar[dd,"\pr_Y"']\ar[rr,"\pr_X"] && X\ar[dd]\\
&&&\\
&Y\ar[rr]&&\tau\end{tikzcd}
\ese
(There is never any need to label a map into a terminal object!)
\item Formulate and prove the dual to part (a).
\een
\ep

\bs
\ben[label=(\alph*)]
\item First note that, since there is a unique morphism $P\to \tau$, the two composites $P\xrightarrow{\, \pr_X\, } X \to \tau$ and $P\xrightarrow{\, \pr_Y\, } Y \to \tau$ are equal, and hence the inner square commutes. Similarly, the outer square also commutes.
Finally, the universal property of the product $P$ implies that there exists a unique morphism $t\cl Z \to P$ making the two triangles commute.
\item Dually, let $\iota$ be an initial object of $\mathcal{C}$ and let $C$ be a coproduct of $X$ and $Y$. Then $C$ solves the problem expressed in the diagram
\bse
\begin{tikzcd}%[column sep=large,row sep=large]
\iota \ar[dd]\ar[rr] && X\ar[dd,"\inj_X"] \ar[dddr,bend left=30,"f"]&\\
&&&\\
Y\ar[rr,"\inj_Y"] \ar[rrrd,bend right=30,"g"]  
 && C \ar[dr,dashed,"\exists!\,t"]&\\
&&& Z
\end{tikzcd}
\ese
Similarly to part (a), the two squares commute because $\iota$ is initial and the existence and uniqueness of $t\cl C\to Z$ follow from the universal property of the coproduct.
\een
\es


\bp
Let $\mathcal{C}$ be a pointed category in which products and coproducts exist for all pairs of objects.
\ben[label=(\alph*)]
\item Give categorical definitions for the `axis' maps $\ax_X\cl X\to X\times Y$ and $\ax_Y\cl Y\to X\times Y$.

{\scshape Note}. The change of notation is due to $\inj_X$ and $\inj_Y$ already being used for the canonical morphisms into the coproduct.
\item In a pointed category, there is a particularly nice morphism $w\cl A\lor B\to A\times B$. Define it in terms of category theory, and check that the diagram
\bse
\begin{tikzcd}[column sep=large,row sep=large]
A\lor B \ar[r,"f\lor g"] \ar[d,"w"']& X \lor Y \ar[d,"w"]\\
A\times B \ar[r,"f\times g"]& X\times Y
\end{tikzcd}
\ese
is commutative for any $f\cl A\to X$ and $g\cl B\to Y$. Why is it necessary for the category to be pointed before you can define $w$?

{\scshape Note}. Here the change of notation is simply to avoid subscripts within subscripts.
\item State (and prove?) the dual statements.
\een
\ep

\bs
\ben[label=(\alph*)]
\item Recall from Exercise 1.34 that there is a bijection
\bse
\mor_{\mathcal{C}}(X,X\times Y)  \cong \mor_{\mathcal{C}}(X,X) \times \mor_{\mathcal{C}}(X, Y) .
\ese
Since $\mathcal{C}$ is pointed, $\mor_{\mathcal{C}}(X, Y)$ is non-empty for all $X,Y\in\mathcal{C}$ as any two objects are connected by the zero map $*$. We can thus define $\ax_X\cl X\to X\times Y$ to be the morphism corresponding to $(\id_X,*)$ under the bijection. Similarly, we define $\ax_Y=(*,\id_Y)$.
\item By the bijection
\bse
\mor_{\mathcal{C}}(A\lor B,A\times B) \cong\mor_{\mathcal{C}}(A,A\times B) \times \mor_{\mathcal{C}}(B,A\times B),
\ese
there is a unique morphism
\bse
w\cl A\lor B\to A\times B
\ese
corresponding to the pair $(\ax_{A},\ax_{B})$. Note that we have
\bi{rCl}
\pr_{A}\circ w\circ\inj_{A} & = & \pr_{A}\circ (\ax_{A},\ax_{B})\circ\inj_{A}\\
& = & \pr_{A}\circ \ax_{A}\\
& = & \pr_{A}\circ(\id_{A},*)\\
& = & \id_{A}
\ei
and, similarly,
\bi{rCl}
\pr_{A}\circ w\circ\inj_{B} & = & *\\
\pr_{B}\circ w\circ\inj_{A} & = & *\\
\pr_{B}\circ w\circ\inj_{B} & = & \id_{B}.
\ei
Thus, we know that every section in the following diagram commutes, except possibly the middle square.
\bse
\begin{tikzcd}[row sep=large,column sep=large]
\ar[ddd,bend right=70,"\id_{A}"'] A \ar[d,"\inj_{A}"']\ar[r,"f"] & X\ar[d,"\inj_{X}"]\ar[ddd,bend left=70,"\id_{X}"]\\
A\lor B \ar[r,"f\lor g"] \ar[d,"w"']& X \lor Y \ar[d,"w"]\\
A\times B \ar[d,"\pr_{A}"']\ar[r,"f\times g"]& X\times Y\ar[d,"\pr_{X}"]\\
A\ar[r,"f"] & X
\end{tikzcd}
\ese
There are similar diagrams in which one or both of the top and bottom maps is replaced by $g\cl B\to Y$, with the appropriate projections and/or injections. Using the commutativity of these diagrams (minus the middle squares), we have
\bse
\pr_X\circ w \circ (f\lor g) \circ \inj_A = \pr_X\circ w \circ\inj_X\circ f=\id_X\circ f=f
\ese
and
\bse
\pr_X\circ (f\times g) \circ w \circ  \inj_A =  f\circ \pr_A\circ w \circ\inj_A=f\circ \id_A=f.
\ese
Similarly, we have
\bse
\pr_X\circ w \circ (f\lor g) \circ \inj_B  = * = \pr_X\circ (f\times g) \circ w \circ  \inj_B
\ese
together with the corresponding results
\bi{rCcCl}
\pr_Y\circ w \circ (f\lor g) \circ \inj_B  & = & g & = & \pr_Y\circ (f\times g) \circ w \circ  \inj_B\\
\pr_Y\circ w \circ (f\lor g) \circ \inj_A  & = & * & = & \pr_Y\circ (f\times g) \circ w \circ  \inj_A.
\ei
By the universal property of coproducts, there are unique maps $r\cl A\lor B\to X$ and $s\cl A\lor B\to Y$ making the diagrams
\bse
\begin{tikzcd}[row sep=large,column sep=large]
A \ar[dr,"f"']\ar[r,"\inj_A"]& A\lor B \ar[d,dashed,"s"]&\ar[l,"\inj_B"'] B\ar[dl,"*"]\\
& X &
\end{tikzcd}
\qquad
\begin{tikzcd}[row sep=large,column sep=large]
A \ar[dr,"*"']\ar[r,"\inj_A"]& A\lor B \ar[d,dashed,"r"]&\ar[l,"\inj_B"'] B\ar[dl,"g"]\\
& Y &
\end{tikzcd}
\ese
commute. Since both $\pr_X\circ w \circ (f\lor g)$ and $\pr_X\circ (f\times g) \circ w$ make the left diagram commute, and both $\pr_Y\circ w \circ (f\lor g)$ and $\pr_Y\circ (f\times g) \circ w$ make the right diagram commute, we must have
\bse
\pr_X\circ w \circ (f\lor g)=s=\pr_X\circ (f\times g) \circ w
\ese
and
\bse
\pr_Y\circ w \circ (f\lor g)=r=\pr_Y\circ (f\times g) \circ w.
\ese
Then, by the universal property of products, there is a unique map $h\cl A\lor B\to X\times Y$ making the diagram
\bse
\begin{tikzcd}[row sep=large,column sep=large]
&A\lor B\ar[dl,"s"']\ar[d,dashed,"h"]\ar[dr,"r"]&\\
X & \ar[l,"\pr_Y"]X\times Y \ar[r,"\pr_X"']& Y
\end{tikzcd}
\ese
commute. Since, as we have just shown, both $w \circ (f\lor g)$ and $(f\times g) \circ w$ make the diagram commute, we must indeed have
\bse
w \circ (f\lor g)=h=(f\times g) \circ w.
\ese

{\scshape Note}. Of course, we could equally well have invoked the universal property of products first to deduce that
\bse
w \circ (f\lor g)\circ \inj_A=(f\times g) \circ w\circ \inj_A
\ese
and
\bse
w \circ (f\lor g)\circ \inj_B=(f\times g) \circ w\circ \inj_B,
\ese
and then used the universal property of coproducts to conclude the proof.

The reason why we need $\mathcal{C}$ to be pointed before we can define $w$ is that, under the bijection of Exercise 1.34, the map $\ax_X\cl X\to X\times Y$ must correspond to a pair of maps, one $X\to X$ and the other $X\to Y$. But, in general, there is no reason why there should be a morphism $X\to Y$ between any two distinct objects $X$ and $Y$.

\item Under the bijection $\mor_{\mathcal{C}}(X\lor Y,Z) \cong\mor_{\mathcal{C}}(X,Z) \times \mor_{\mathcal{C}}(Y, Z)$, setting $Z=X$ and $Z=Y$, we define the `collapse' maps $\co_X\cl X\lor Y\to X$ and $\co_Y\cl X\lor Y\to Y$ by $\co_X=(\id_X,*)$ and $\co_Y=(*,\id_Y)$, respectively.
\een
\es

\bx
In what sense do the maps $w$ constitute a natural transformation?
\ex

\bs
\es

\bp
Show that for any $f\cl X \to Y$ in a pointed category $\mathcal{C}$ and $*\cl W\to X$, then $f\circ * = *\cl W\to Y$. Also show that if $*\cl Y\to Z$, then $*\circ f=*\cl X\to Z$. Conclude that the functors $\mor_{\mathcal{C}}(-,Y)$ and $\mor_{\mathcal{C}}(A,-)$ take their values in the category of pointed sets and pointed maps.
\ep

\bs
The morphism $f\circ *$ factors as $W\xrightarrow{*\, } 0 \xrightarrow{*\, } X\xrightarrow{f\, } Y$. Since there is a unique morphism $0\to Y$, namely $*$, we have $0 \xrightarrow{*\, } X\xrightarrow{f\, } Y=0 \xrightarrow{*\, }Y$. Therefore,
\bse
f\circ * = W\xrightarrow{*\, } 0 \xrightarrow{*\, } X\xrightarrow{f\, } Y=W\xrightarrow{*\, } 0 \xrightarrow{*\, }  Y = *.
\ese
Similarly, for the composite $*\circ f=*\cl X\to Z$, we have
\bse
*\circ f = X\xrightarrow{f\, } Y\xrightarrow{*\, } 0 \xrightarrow{*\, } Z=X\xrightarrow{*\, } 0 \xrightarrow{*\, }  Z = *.
\ese
For any $X\in\mathcal{C}$, the sets $\mor_{\mathcal{C}}(X,Y)$ and $\mor_{\mathcal{C}}(A,X)$ have basepoints $*\cl X\to Y$ and $*\cl A\to Y$, respectively and, for any morphism $f\cl X\to B$ in $\mathcal{C}$, the induced morphisms $f^*\cl\mor_{\mathcal{C}}(B,Y)\to \mor_{\mathcal{C}}(X,Y)$ and $f_*\cl\mor_{\mathcal{C}}(A,X)\to\mor_{\mathcal{C}}(X,B)$ are given by pre- and post-composition with $f$, respectively. As we have just shown, both of there preserve $*$, so they are pointed.
\es

\bx
\ben[label=(\alph*)]
\item Show that the trivial group $\{1\}$ is simultaneously initial and terminal in the category $G$ of groups and homomorphisms. Show that the vector space $0$ is simultaneously initial and terminal in the category of vector spaces (over $\R$, if you like) and linear transformations.
\item Show that in the category $\mathbf{AbGrp}$ of abelian groups and homomorphisms, the map $w \cl G \lor G \to G \times H$ is an isomorphism for any $G$ and $H$.
Also show that the analogous statement is true in the category of vector spaces and linear transformations.
\een
\ex

\bs
\ben[label=(\alph*)]
\item
\item
\een
\es

\bp
\ben[label=(\alph*)]
\item Show that, in any category $\mathcal{C}$, there is a natural bijection between the morphism set $\mor_{\mathcal{C}}(X_1 \sqcup X_2, Y_1 \times Y_2)$ and the set $M$ of all matrices
\bse
\begin{bmatrix}f_{11}& f_{12}\\
f_{21} & f_{22}\end{bmatrix}
\ese
with $f_{ij} \in \mor_{\mathcal{C}}(X_j, Y_i)$.
\item Now suppose that $\mathcal{C}$ is a pointed category in which the canonical map $w\cl X\lor Y \to X\times Y$ is an isomorphism for each pair of objects $X, Y \in \mathcal{C}$. Show that composition
\bse
\rule{-1cm}{0cm}
\begin{tikzcd}[row sep=large,column sep=large]
\mor_{\mathcal{C}}(Y_1 \sqcup Y_2, Z_1 \times Z_2) \times \mor_{\mathcal{C}}(X_1 \sqcup X_2, Y_1 \times Y_2) \ar[d,"\cong"] \ar[r] & \mor_{\mathcal{C}}(X_1 \sqcup X_2, Z_1 \times Z_2)\\
\mor_{\mathcal{C}}(Y_1 \times Y_2, Z_1 \times Z_2) \times \mor_{\mathcal{C}}(X_1 \sqcup X_2, Y_1 \times Y_2) \ar[ur,bend right=10]&
\end{tikzcd}
\ese
corresponds to matrix multiplication in $M$.
\item Show that linear transformations $\R^2 \to \R^2$ are in one-to-one correspondence with $2 \times 2$ matrices with real entries.
\een
\ep

\bs
\ben[label=(\alph*)]
\item
\item
\item
\een
\es

\section{Group and Cogroup Objects}

\bx
Are group objects domain-type or target-type gadgets? Suppose $G$ is a group object in $\mathcal{C}$. What conditions must you impose on a functor $F \cl \mathcal{C} \to \mathcal{D}$ in order to conclude that $F(G) \in \mathcal{D}$ is also a group object?
\ex

\bs
\es

\bx
\ben[label=(\alph*)]
\item Check that, in the category of pointed sets, a group object is just an ordinary group.
\item Show that a group $G \in \mathbf{Grp}$ is a group object if, and only if, $G$ is abelian.
\item Write $\GL_n(\R)$ to denote the set of all $n \times n$ invertible matrices. It is a subset of $\R^{n^2}$, so we can give it the subspace topology. Show that matrix multiplication makes $\GL_n(\R)$ into a group object in the category of pointed topological spaces.
\een
\ex

\bs
\ben[label=(\alph*)]
\item
\item
\item
\een
\es

\bx
You know that in the category of pointed sets and their maps, the inverse map $\nu$ for a group object $G$ is uniquely determined by its multiplication $\mu$. Prove that this is true for group objects in any category.
\ex

\bs
\es

\bp
Let $G$ be a group object in a pointed category $\mathcal{C}$.
\ben[label=(\alph*)]
\item Show that the composite map $M$ in the diagram
\bse
\begin{tikzcd}[row sep=large,column sep=large]
\mor_{\mathcal{C}}(X,G) \times \mor_{\mathcal{C}}(X,G) \ar[r,"M"] \ar[d,"\cong"']& \mor_{\mathcal{C}}(X,G)\\
\mor_{\mathcal{C}}(X,G \times G)\ar[ru,bend right=10,"\mu_*"'] &
\end{tikzcd}
\ese
makes $\mor_{\mathcal{C}}(X,G)$ into a group object in the category of pointed sets (i.e., $\mor_{\mathcal{C}}(X,G)$ is a group with multiplication $M$).
\item Draw a diagram that shows all the maps involved in the definition of the product of $\alpha,\beta\in\mor_{\mathcal{C}}(X,G)$ and how they fit together. (In other words, write down explicitly what $\alpha\cdot\beta$ is.)
\item Let $f \cl X \to Y$ be a morphism in $\mathcal{C}$. Show that $f_*\cl \mor_{\mathcal{C}}(Y,G) \to \mor_{\mathcal{C}}(X,G)$ is a group homomorphism.

{\scshape Hint}. Use Problem 1.37 and part (b).
\een
\ep

\bs
\ben[label=(\alph*)]
\item
\item
\item
\een
\es

\addtocounter{exercise}{1}
\bx
Let $G$ be a group object in a category $\mathcal{C}$. Work out the product $\pr_1\cdot \pr_2\in \mor_{\mathcal{C}}(G \times G,G)$. You should be able to express it as a specific map you already know.
\ex

\bs
\es

\bp
Write down the definition of a cogroup object. What is a cogroup object in the category of groups and homomorphisms? What about abelian groups and homomorphisms? What is a cogroup object in the category of pointed sets? What if you replace `cogroup' with `comonoid'?
\ep

\bs
\es

\addtocounter{exercise}{1}
\bp
Prove Theorem 1.58: If $C$ is a cogroup object in a pointed category $\mathcal{C}$, then the covariant functor $G(Y) = \mor_{\mathcal{C}}(C,Y)$ takes its values in the category of groups and homomorphisms.
\ep

\bs
\es

\bp
Suppose $C$ is a cogroup object and $G$ is a group object. Then $\mor_{\mathcal{C}}(C,G)$ is a group because $C$ is a cogroup object---we'll write $\alpha\mathbin{\spadesuit} \beta$ for this product; and $\mor_{\mathcal{C}}(C,G)$ is a group because $G$ is a group object---we'll write $\alpha\mathbin{\heartsuit}\beta$ for this product. Show that the products $\mathbin{\spadesuit}$ and $\mathbin{\heartsuit}$ are the same. That is, show that for any $f, g \in\mor_{\mathcal{C}}(C,G)$, $f\mathbin{\spadesuit} g=f\mathbin{\heartsuit} g$.

{\scshape Hint}. Write down the compositions which define $f\mathbin{\spadesuit} g$ and $f\mathbin{\heartsuit} g$ in a single commutative diagram. Use Problem 1.46.
\ep

\bs
\es

\section{Homomorphisms}

\bx
Show that when $\mathcal{C}$ is the category of pointed sets, a homomorphism of group objects is just the same as a homomorphism of groups.
\ex

\bs
\es

\bx
Show that if $f \cl G \to H$ is a homomorphism of group objects in $\mathcal{C}$, then $f$ preserves inverses, in the sense that the diagram
\bse
\begin{tikzcd}[row sep=large,column sep=large]
G \ar[r,"f"] \ar[d,"\nu_G"] & H \ar[d,"\nu_H"] \\
G\ar[r,"f"] & H
\end{tikzcd}
\ese
is commutative.
\ex

\bs
\es

\addtocounter{exercise}{1}

\bp
Prove Theorem 1.63: If $f \cl G \to H$ is a homomorphism of group objects in the pointed category $\mathcal{C}$, then the induced map
\bse
f_* \cl \mor_{\mathcal{C}}(X,G) \longrightarrow \mor_{\mathcal{C}}(X,H)
\ese
is a homomorphism of groups for every $X \in\mathcal{C}$.

Is the converse true?
\ep

\bs
\es

\bp
Define homomorphisms of cogroup objects, and prove that they induce group homomorphisms on mapping sets.
\ep

\bs
\es

\section{Abelian Groups and Cogroups}

\bp
Write down a categorical description of the twist map $T\cl X\sqcup Y\to X\sqcup Y$. Also define the twist map $T \cl X \times Y \to Y \times X$ for products. Show that both twist maps are equivalences.
\ep

\bs
\es

\bp
Show that $C$ is a cocommutative cogroup if, and only if, $\mor_{\mathcal{C}}(C,Y)$ is an abelian group for every $Y$.
\ep

\bs
\es

\bp
Dualise this discussion: define a commutative group object, and verify that $\mor_{\mathcal{C}}(X,G)$ is abelian if, and only if, $G$ is such an object.
\ep

\bs
\es

\bp
\ben[label=(\alph*)]
\item Show that $\mu_{G\times H}$ and $\nu_{G\times H}$ make $G \times H$ into a group object in $\mathcal{C}$.
\item Show that $\inj_1 \cl G \to G\times H$ and $\inj_2 \cl H \to G \times H$ are homomorphisms.
\item Show that $\pr_1 \cl G\times H \to G$ and $\pr_2 \cl G\times H \to H$ are homomorphisms.
\een
\ep

\bs
\ben[label=(\alph*)]
\item
\item
\item
\een
\es

\bp
Show that a group object is commutative if, and only if, the multiplication $\mu\cl G \times G \to G$ is a homomorphism. What is the dual statement?
\ep

\bs
\es

\section{Adjoint Functors}

\bx
Let $f \cl X \to X'$ and $g \cl Y \to Y'$. Write down the diagrams which must commute in order for $\Phi$ to be a natural transformation.
\ex

\bs
\es

\bp
Let $L$ and $R$ be adjoint.
\ben[label=(\alph*)]
\item Show that if one of the squares
\bse
\begin{tikzcd}[row sep=large,column sep=huge]
LA \ar[r,"\alpha"]\ar[d,"Lf"']& X\ar[d,"g"]\\
LB\ar[r,"\beta"] & Y
\end{tikzcd}
\qquad \text{ and } \qquad
\begin{tikzcd}[row sep=large,column sep=huge]
A \ar[r,"\widehat\alpha"]\ar[d,"f"']& RX\ar[d,"Rg"]\\
B\ar[r,"\widehat\beta"] & RY
\end{tikzcd}
\ese
commutes, then so does the other one.
\item Let $F,G$ be functors. Show that $\Phi \cl LF \to G$ is a natural transformation if, and only if, $\widehat\Phi\cl F \to RG$ is a natural transformation.
\een
\ep

\bs
\ben[label=(\alph*)]
\item
\item
\een
\es

\bx
If $X$ is a set, then we can form the free abelian group $F(X) = \bigoplus_{x\in X} \Z$. On the other hand, if $G$ is an abelian group, then we can forget the group structure of $G$ and just remember the underlying set $S(G)$.
\ben[label=(\alph*)]
\item Show that the functors $F \cl \mathbf{Sets} \to \mathbf{Grp}$ and $S \cl \mathbf{Grp} \to \mathbf{Sets}$ are adjoint to one another. Which is the left adjoint and which is the right adjoint?
\item Use the same scheme to express other `free objects' that you know about in terms of adjoints.
\een
\ex

\bs
\ben[label=(\alph*)]
\item
\item
\een
\es

\bp
Show that there is a commutative diagram
\bse
\begin{tikzcd}[row sep=large,column sep=huge]
\mor_{\mathcal{C}}(X, Y ) \ar[d,equal] \ar[r,"L"] & \mor_{\mathcal{C}}(L(X), L(Y ))\ar[d,"\Phi","\cong"']\\
\mor_{\mathcal{C}}(X, Y )\ar[r,"\sigma_*"]& \mor_{\mathcal{C}}(X,RL(Y ))
\end{tikzcd}
\ese
where $\sigma\cl Y\to RL(Y)$ is $\Phi_{LY,LY}(\id_{LY})$.
\ep

\bs
\es

\bx
Problem 1.74 shows that the maps $L$ and $\sigma_*$ are equivalent maps. In Section 1.2 we defined what equivalence means in categorical terms---what category are we working in here?
\ex

\bs
\es

\bp
Dualize the previous problem.
\ep

\bs
\es

\bp
Let $L \cl \mathcal{C} \to \mathcal{D}$ and $R \cl\mathcal{D} \to \mathcal{C}$ be a pair of adjoint functors. Show that there are natural maps $X \to RLX$ and $LRX \to X$ and that $RX$ is naturally a retract of $RLRX$.
\ep

\bs
\es

\begin{extrap}
Show that if $L, L' \cl \mathcal{C} \to \mathcal{D}$ are both left adjoints to $R \cl \mathcal{D}\to\mathcal{C}$, then there is a natural isomorphism $L \to L'$. So adjoints are unique up to isomorphism.
\end{extrap}

\bs
For $A\in \mathcal{C}$ and $B\in\mathcal{D}$, denote by $\Phi_{A,B}\cl\mor_{\mathcal{D}}(LA,B)\to\mor_{\mathcal{C}}(A,RB)$ and $\Psi_{A,B}\cl\mor_{\mathcal{D}}(L'A,B)\to\mor_{\mathcal{C}}(A,RB)$ the natural bijections corresponding to the adjunctions $L\dashv R$ and $L'\dashv R$, respectively.

For each $A\in \mathcal{C}$, define $\xi_A=(\Psi_{A,LA})^{-1}(\Phi_{A,LA}(\id_{LA}))$, i.e.\ $\xi_A$ is the image of $\id_{LA}$ under the composite
\bse
\mor_{\mathcal{D}}(LA,LA)\xrightarrow{\ \Phi_{A,LA} \ }\mor_{\mathcal{C}}(A,RLA)\xrightarrow{\ (\Psi_{A,LA})^{-1}\ }\mor_{\mathcal{D}}(L'A,LA).
\ese
Then $\xi\cl L'\to L$ is a natural isomorphism. To see this, let $f\cl A\to B$ be a morphism in $\mathcal{C}$. We seek to show that the diagram
\bse
\begin{tikzcd}[row sep=large,column sep=large]
L'A \ar[r,"\xi_A"]\ar[d,"L'f"] & LA\ar[d,"Lf"]\\
L'B\ar[r,"\xi_B"]&LB
\end{tikzcd}
\ese
commutes. Consider the following diagram:
\bse
\begin{tikzcd}[row sep=huge,column sep=4.5em]
\mor_{\mathcal{D}}(LA,LA) \ar[r,"\Phi_{A,LA}"]\ar[d,"Lf\circ-"]& \mor_{\mathcal{C}}(A,RLA)\ar[r,"(\Psi_{A,LA})^{-1}"] \ar[d,"RLf\circ-"]& \mor_{\mathcal{D}}(L'A,LA)\ar[d,"Lf\circ-"]\\
\mor_{\mathcal{D}}(LA,LB) \ar[r,"\Phi_{A,LB}"]& \mor_{\mathcal{C}}(A,RLB)\ar[r,"(\Psi_{A,LB})^{-1}"] & \mor_{\mathcal{D}}(L'A,LB)
\end{tikzcd}
\ese
The internal squares commute by the assumption that we have adjunctions $L\dashv R$ and $L'\dashv R$. Hence, so does the outer rectangle. Following $\id_{LA}$ around the diagram, we obtain
\bi{rCl}
Lf\circ\xi_A & = & (\Psi_{A,LB})^{-1}(\Phi_{A,LB}(Lf\circ\id_{LA}))\\
& = & (\Psi_{A,LB})^{-1}(\Phi_{A,LB}(Lf\circ L(\id_{A})))\\
& = & (\Psi_{A,LB})^{-1}(\Phi_{A,LB}(L(f\circ\id_{A})))\\
& = & (\Psi_{A,LB})^{-1}(\Phi_{A,LB}(L(f))).
\ei
Similarly, by considering the diagram
\bse
\begin{tikzcd}[row sep=huge,column sep=4.5em]
\mor_{\mathcal{D}}(LB,LB) \ar[r,"\Phi_{B,LB}"]\ar[d,"-\circ Lf"]& \mor_{\mathcal{C}}(B,RLB)\ar[r,"(\Psi_{B,LB})^{-1}"] \ar[d,"-\circ f"]& \mor_{\mathcal{D}}(L'B,LB)\ar[d,"-\circ L'f"]\\
\mor_{\mathcal{D}}(LA,LB) \ar[r,"\Phi_{A,LB}"]& \mor_{\mathcal{C}}(A,RLB)\ar[r,"(\Psi_{A,LB})^{-1}"] & \mor_{\mathcal{D}}(L'A,LB)
\end{tikzcd}
\ese
we find
\bi{rCl}
\xi_B\circ L'f & = & (\Psi_{A,LB})^{-1}(\Phi_{A,LB}(\id_{LB}\circ Lf))\\
& = & (\Psi_{A,LB})^{-1}(\Phi_{A,LB}(Lf)).
\ei
So $Lf\circ\xi_A=\xi_B\circ L'f$, i.e.\ $\xi$ is a natural transformation.

Finally, for each $A\in \mathcal{C}$, define $\lambda_A=(\Phi_{A,L'A})^{-1}(\Psi_{A,L'A}(\id_{L'A}))$, i.e.\ $\lambda_A$ is the image of $\id_{L'A}$ under the composite
\bse
\mor_{\mathcal{D}}(L'A,L'A)\xrightarrow{\ \Psi_{A,L'A} \ }\mor_{\mathcal{C}}(A,RL'A)\xrightarrow{\ (\Phi_{A,L'A})^{-1}\ }\mor_{\mathcal{D}}(LA,L'A).
\ese
By the naturality of the bijections $\Phi$ and $\Psi$, the diagram
\bse
\begin{tikzcd}[row sep=huge,column sep=4.5em]
\mor_{\mathcal{D}}(L'A,L'A) \ar[r,"\Psi_{A,L'A}"]\ar[d,"\xi_A\circ-"]& \mor_{\mathcal{C}}(A,RL'A)\ar[r,"(\Phi_{A,L'A})^{-1}"] \ar[d," R\xi_A\circ-"]& \mor_{\mathcal{D}}(LA,L'A)\ar[d,"\xi_A\circ-"]\\
\mor_{\mathcal{D}}(L'A,LA) \ar[r,"\Psi_{A,LA}"]& \mor_{\mathcal{C}}(A,RLA)\ar[r,"(\Phi_{A,LA})^{-1}"] & \mor_{\mathcal{D}}(LA,LA)
\end{tikzcd}
\ese

commutes. Chasing $\id_{L'A}$ around the diagram, we obtain
\bi{rCl}
\xi_A\circ\lambda_A & = & (\Phi_{A,LA})^{-1}(\Psi_{A,LA}(\xi_A\circ \id_{L'A}))\\
& = & (\Phi_{A,LA})^{-1}(\Psi_{A,LA}(\xi_A))\\
& = &  (\Phi_{A,LA})^{-1}(\Psi_{A,LA}(\Psi_{A,LA})^{-1}(\Phi_{A,LA}(\id_{LA})))\\
& = &  (\Phi_{A,LA})^{-1}(\Phi_{A,LA}(\id_{LA}))\\
& = &\id_{LA}
\ei
and, similarly, $\lambda_A\circ\xi_A=\id_{L'A}$ for each $A\in \mathcal{C}$. So $\xi_A$ is an isomorphism for each $A\in\mathcal{C}$, and thus $\xi\cl L'\to L$ is a natural isomorphism with inverse $\lambda$.
\es





































\chapter{Limits and Colimits}

\section{Diagrams and Their Shapes}

\bx
Determine the shape of the diagram\footnote{In the sense of Section 1.1.}
\bse
\begin{tikzcd}[row sep=large,column sep=large]
A \ar[r]\ar[dr] & B\ar[d]\\
& C
\end{tikzcd}
\ese
if the diagram is
\ben[label=(\alph*)]
\item \textbf{guaranteed} to be commutative;
\item \textbf{not necessarily} noncommutative.
\een 
\ex

\bs
\ben[label=(\alph*)]
\item  
\item 
\een
\es

\bx
Let $G$ be a group, and consider it as a category with one object, as in Exercise 1.3. Interpret `diagram with shape $G$' in the following categories.
\ben[label=(\alph*)]
\item  The category of sets and maps.
\item The category of groups and homomorphisms.
\item The category of topological spaces and continuous maps.
\een
\ex

\bs
\ben[label=(\alph*)]
\item  
\item
\item 
\een
\es

\bx
Consider the group $G = \Z/2$ as a category with one object and two morphisms. Is $G$ a finite shape category?
\ex

\bs
\es

\section{Limits and Colimits}

\bx
Suppose that the empty diagram $\vn$ has a limit. Show that it is a terminal object in $\mathcal{C}$. Find a way to view a product as the limit of a diagram.
\ex

\bs
\es

\bx
\ben[label=(\alph*)]
\item Construct an example of a category and a diagram which has no limit.
\item A discrete category is one in which the only morphisms are identities. What is the limit of a diagram whose shape is discrete?
\een
\ex

\bs
\ben[label=(\alph*)]
\item  
\item 
\een
\es

\bp
Show that any two limits of a given diagram $F \cl \mathcal{I} \to \mathcal{C}$ are equivalent in $\mathcal{C}$. Also prove the dual statement about the uniqueness of colimits.

{\scshape Hint.} Let $P$ and $Q$ be two limits of $F$; use the universal property of the limit to find maps $P \to Q$ and $Q \to P$. Alternatively, construct a natural isomorphism between the functors $\mor_{\mathcal{C}}( - , P)$ and $\mor_{\mathcal{C}}( -,Q)$.
\ep

\bs
\es

\bx
Define the opposite $F^\op \cl \mathcal{I}^\op \to \mathcal{C}^\op$ of a diagram $F \cl\mathcal{I} \to\mathcal{C}$. Compare the limit and colimit of $F^\op$ to the limit and colimit of $F$.
\ex

\bs
\es

\bx
Show that the colimit of the empty diagram $\vn \to \mathcal{C}$ is an initial object in $\mathcal{C}$. Show that the limit of the identity $\id_\mathcal{C}$ is an initial object in $\mathcal{C}$.
\ex

\bs
\es

\bx
Suppose $\mathcal{I}$ has an initial object, $\vn$, and let $F \cl \mathcal{I} \to \mathcal{C}$. Show that $F$ has a limit. State and prove the dual result.
\ex

\bs
\es

\bx
What is the colimit of a diagram whose shape is discrete?
\ex

\bs
\es



\section{Naturality of Limits and Colimits}

\section{Special Kinds of Limits and Colimits}

\bp
Suppose that
\bse
\begin{tikzcd}[column sep=large,row sep=large]
P \ar[d,"i"']\ar[r,"p"] & X\ar[d,"j"]\\
Z\ar[r,"z"]&Y\end{tikzcd}
\ese
is a pullback square. Show that if $j$ is an equivalence, then so is $i$.

{\scshape Hint}. Use the map $j^{-1}\circ z$ to find a map $K\cl Z\to P$.
\ep

\bs
Consider the diagram
\bse
\begin{tikzcd}%[column sep=large,row sep=large]
Z\ar[dr,dashed,"K"]\ar[rrrd,bend left=30,"j^{-1}\circ z"] \ar[dddr,bend right=30,"\id_Z"'] &&\\
&P \ar[dd,"i"']\ar[rr,"p"] && X\ar[dd,"j"]\\
&&&\\
&Z\ar[rr,"z"]&&Y\end{tikzcd}
\ese
and note that, since $j\circ(j^{-1}\circ z)=z\circ \id_Z$, there exists a unique $K\cl Z\to P$ such that the two triangles commute, i.e. $p\circ K=j^{-1}\circ z$ and $i\circ K=\id_Z$. Now consider the diagram
\bse
\begin{tikzcd}%[column sep=large,row sep=large]
P\ar[dr,dashed]\ar[rrrd,bend left=30,"p"] \ar[dddr,bend right=30,"i"'] &&\\
&P \ar[dd,"i"']\ar[rr,"p"] && X\ar[dd,"j"]\\
&&&\\
&Z\ar[rr,"z"]&&Y\end{tikzcd}
\ese
The outer square commutes since it is the same as the inner square. Hence there is a unique morphism $P\to P$ making the two triangles commute. We obviously have $p\circ \id_P=p$ and $i\circ \id_P=i$, but also
\bse
p\circ(K\circ i)=(p\circ K)\circ i=j^{-1}\circ z \circ i=j^{-1}\circ j \circ p=p
\ese
and $i\circ(K\circ i)=(i\circ K)\circ i=i$. Thus $K\circ i=\id_P$.
\es

\bx
Suppose that $\mathcal{C}$ has a terminal object $\tau$. Show that products of $X$ and $Y$ are the same as pullbacks for the diagram $X\to \tau \leftarrow Y$.
\ex

\bs
\begin{itemize}
\item[$(\Rightarrow)$]
This is done in Problem 1.45 (a).
\item[$(\Leftarrow)$]
Let $(Z,p,q)$ be a pullback of the diagram $X\to\tau\leftarrow Y$. Consider arbitrary morphisms $f\cl W\to X$ and $g\cl W\to Y$. Since $\tau$ is terminal, there is a unique morphism $W\to \tau$. Therefore, the two composites $W\xrightarrow{\,f\,}X\to \tau$ and $W\xrightarrow{\,g\,}Y\to \tau$ are equal. By the universal property of pullbacks, there is a unique morphism $W\to Z$ making the diagram 
\bse
\begin{tikzcd}
W \ar[dddr,bend right=30,"g"]\ar[drrr,bend left=30,"f"]\ar[dr,dashed]&&&\\
& Z \ar[rr,"p"]\ar[dd,"q"]&& X\ar[dd]\\
&&&\\
&Y \ar[rr]&& \tau
\end{tikzcd}
\ese
commute. Thus, $(Z,p,q)$ is a product of $X$ and $Y$. 
\end{itemize}
\es

\bx
Determine the pullback of the diagram
\bse
\begin{tikzcd}
\{1\}\ar[r] & B & \ar[l,"f"'] A
\end{tikzcd}
\ese
in the category of groups and homomorphisms.
\ex

\bs
Consider the subgroup $\ker (f)$ of $A$ and let $i\cl \ker (f)\hookrightarrow A$ be the inclusion. Then, denoting all identities as $1$, we have $f(i(a))=1$ for all $a\in \ker(f)$. So the inner square commutes. (Note that the homomorphisms into and out of $\{1\}$ are all uniquely determined as $\{1\}$ is a zero object.)
\bse
\begin{tikzcd}
Q \ar[rrrd,bend left=30,"h"] \ar[dddr,bend right=30] \ar[dr,shift left,"j"]\ar[dr,shift right,"j'"']&&&\\
& \ker (f) \ar[rr,hook,"i"]\ar[dd] && A\ar[dd,"f"]\\
&&&\\
& \{1\} \ar[rr] && B
\end{tikzcd}
\ese
Let $h\cl Q\to A$ be such that the outer diagram commutes, i.e.\ $f(h(q))=1$ for all $q\in Q$. Then $h(q)\in\ker(f)$ for all $q\in Q$, so $h$ factors through the inclusion of the kernel, i.e.\ $h=ij$, with $j\cl Q\to \ker (f)$. 

Finally, suppose that $j'\cl Q\to \ker(f)$ is another homomorphism making the triangles commute. Then
\bse
j'(q)=i(j'(q))=h(q)=i(j(q))=j(q)
\ese
for all $q\in Q$, so $j=j'$. Thus $\ker(f)$ is the pullback of $\{1\}\to B \xleftarrow{\ f}A$.
\es

\bp
State and prove the dual of Problem 2.21.
\ep

\bs
Suppose that
\bse
\begin{tikzcd}[row sep=large,column sep=large]
A \ar[r,"i"] \ar[d,"f"]& B\ar[d,"g"]\\
C \ar[r,"j"] & D
\end{tikzcd}
\ese
is a pushout square. Then, if $f$ is an equivalence, so is $g$. Indeed, we have a diagram
\bse
\begin{tikzcd}%[row sep=large,column sep=large]
A \ar[rr,"i"] \ar[dd,"f"]&& B\ar[dd,"g"]\ar[dddr,bend left=30,"\id_B"]&\\
&&&\\
C \ar[rrrd,bend right=30,"i\circ f^{-1}"']\ar[rr,"j"] && D\ar[dr,dashed,"k"]&\\
&&&B
\end{tikzcd}
\ese
Since $(i\circ f^{-1})\circ f=i=\id_B\circ i$, by the universal property of pushouts, there is a unique morphism $k\cl D\to B$ such that $k\circ g=\id_B$ and $k\circ j=i\circ f^{-1}$. Now consider
\bse
\begin{tikzcd}%[row sep=large,column sep=large]
A \ar[rr,"i"] \ar[dd,"f"]&& B\ar[dd,"g"]\ar[dddr,bend left=30,"g"]&\\
&&&\\
C \ar[rrrd,bend right=30,"j"']\ar[rr,"j"] && D\ar[dr,dashed]&\\
&&&D
\end{tikzcd}
\ese
The outer square commutes (as it is the same as the inner square) and hence there exists a unique morphism $D\to D$ making the two triangles commute. We obviously have $\id_D\circ g= g$ and $\id_D\circ j= j$, but also
\bse
(g\circ k)\circ j=g\circ (k\circ j)=g\circ i\circ f^{-1}=j\circ f\circ f^{-1}=j
\ese
and $(g\circ k)\circ g=g\circ(k\circ g)=g\circ \id_B=g$, so $g\circ k=\id_D$.  
\es

\bx
Work in the category $\mathbf{Sets}$ of all sets and all functions between sets. Suppose that $A \subseteq B$ and $A\subseteq C$ and that $i \cl A \to C$ and $p \cl A \to B$ are inclusion functions.
\ben[label=(\alph*)]
\item Determine the pushout of the diagram $B\xleftarrow{\ p} A\xrightarrow{i\ } C$.
\item Let $*$ denote a one-point set. Determine the pushout of $B\xleftarrow{\ p} A\xrightarrow{*\ }*$.
\een
\ex

\bs
\es

\bp
Let $X$ be a topological space, and suppose $X = A \cup B$, where $A,B \subseteq X$ are closed subspaces. Show that the diagram
\bse
\begin{tikzcd}[row sep=large,column sep=large]
A \cap B \ar[r]\ar[d] & A\ar[d]\\
B\ar[r]& X
\end{tikzcd}
\ese
is a pushout square.
\ep

\bs
\es

\bx
Suppose $\mathcal{C}$ has an initial object $\iota$, and let $X, Y \in \mathcal{C}$. Show that pushouts of diagrams of the form $X \leftarrow \iota \to Y$ are the same as coproducts $X \sqcup Y$.
\ex

\bs
\begin{itemize}
\item[$(\Rightarrow)$]
This is done in Problem 1.45 (b).
\item[$(\Leftarrow)$]
Let $(Z,i,j)$ be a pushout of the diagram $X \leftarrow \iota \to Y$. Consider arbitrary morphisms $f\cl X\to W$ and $g\cl Y\to W$. Since $\iota$ is initial, there is a unique morphism $\iota\to W$. Therefore, the two composites $\iota \to X\xrightarrow{\,f\,}W$ and $\iota \to Y\xrightarrow{\,g\,}W$ are equal. By the universal property of pushouts, there is a unique morphism $Z\to W$ making the diagram 
\bse
\begin{tikzcd}%[row sep=large,column sep=large]
\iota \ar[rr] \ar[dd]&& X\ar[dd,"i"]\ar[dddr,bend left=30,"f"]&\\
&&&\\
Y \ar[rrrd,bend right=30,"g"']\ar[rr,"j"] && Z\ar[dr,dashed]&\\
&&&W
\end{tikzcd}
\ese
commute. Thus, $(Z,i,j)$ is a coproduct of $X$ and $Y$. 
\end{itemize}
\es

\bx
\ben[label=(\alph*)]
\item Consider the category whose objects are the integers $1, 2, 3, \ldots$ and with arrows given by divisibility. Give number-theoretical descriptions of pushouts in this category.
\item Repeat (a) but using the category whose objects are real numbers and whose morphisms correspond to inequalities $x \leq y$.
\een
\ex

\bs
\ben[label=(\alph*)]
\item 
\item 
\een
\es

\bx
Determine the pushout of the diagram $A\xleftarrow{\ f} B\to\{1\}$ in the category of groups and homomorphisms. Compare with Exercise 2.25; what should the object you constructed here be called?
\ex

\bs
Given a subgroup $S$ of a group $G$, denote by $N(S)$ the \emph{normal closure} of $S$ in $G$, i.e.\ the intersection of all normal subgroups of $G$ which contain $S$. So $N(S)$ is the smallest normal subgroup of $G$ which contains $S$. Explicitly,
\bse
N(S)=\{g^{-1}sg \mid g \in G \text{ and } s \in S\}.
\ese
Consider the subgroup $\im (f)$ of $B$, take the quotient $B/N(\im(f))$ and let $q\cl B \to B/N(\im(f))$ be the quotient homomorphism. Then, for all $a\in A$, we have $f(a)\in \im(f)$, so $q(f(a))=N(\im(f))$ and thus, the inner square commutes. (Note that the homomorphisms into and out of $\{1\}$ are all uniquely determined as $\{1\}$ is a zero object.)
\bse
\begin{tikzcd}%[row sep=large,column sep=large]
A \ar[rr,"f"] \ar[dd]&& B\ar[dd,"q"]\ar[dddr,bend left=30,"h"]&\\
&&&\\
\{1\} \ar[rrrd,bend right=30]\ar[rr] && {\displaystyle \frac{B}{N(\im(f))} } \ar[dr,shift left,"j"]\ar[dr,shift right,"j'"']&\\
&&&Z
\end{tikzcd}
\ese
Let $h\cl B\to Z$ be such that the outer diagram commutes, i.e., denoting all identities as $1$, $h(f(a))=1$ for all $a\in A$. Then $h(b)=1$ for all $b\in \im(f)$. Moreover, if $b\notin \im(f)$ but $b\in N(\im(f))$, then $b=g^{-1}b_0 g$ for some $g\in B$ and $b_0\in \im(f)$. Therefore
\bse
h(b)=h(g^{-1}b_0 g)=h(g^{-1})h(b_0)h( g)=h(g)^{-1}h(g)=1,
\ese
so $h$ factors through $q$ as $h=jq$, with $j\cl B/N(\im(f))\to Z$ defined by
\bse
j(bN(\im(f)))=h(b).
\ese

Finally, suppose that $j'\cl B/N(\im(f))\to Z$ is another homomorphism making the triangles commute. Then
\bse
j'(bN(\im(f)))=j'(q(b))=h(b)=j(bN(\im(f)))
\ese
for all $b\in B$, so $j=j'$. Thus $B/N(\im(f))$ is the pushout of $A\xleftarrow{\ f} B\to\{1\}$. 

The object $B/N(\im(f))$ is called the \emph{cokernel} of $f\cl A\to B$, and is denoted by $\coker(f)$, as it satisfies the dual of the universal property defining $\ker(f)$. In $\mathbf{AbGrp}$, we have $\coker(f)=B/\im(f)$ for any $f\cl A\to B$, as every subgroup of an abelian group is normal (so $N(\im(f))=\im(f)$).
\es

\bp
Define the dual notion of \textbf{coequalizer}, and compare coequalizers with pushouts.
\ep

\bs
\es

\section{Formal Properties of Pushout and Pullback Squares}

\bp
Consider the diagram
\bse
\begin{tikzcd}[row sep=large,column sep=large]
A \ar[r,"i"] \ar[d,"f"]& B\ar[d,"g"]\\
C \ar[r,"j"] & D
\end{tikzcd}
\ese
\ben[label=(\alph*)]
\item Suppose the diagram is a pushout and that $f$ is an equivalence. Show that $g$ is also an equivalence.
\item Suppose $f$ and $g$ are both equivalences. Show that the square is a pushout.
\item State and prove the duals of (a) and (b).
\een
\ep

\bs
\ben[label=(\alph*)]
\item This is done in Problem 2.27.
\item {\scshape Note}. The statement in (b) is not true without also assuming that the square is commutative. Indeed, consider the following square in, say, $\mathbf{Sets}$.
\bse
\begin{tikzcd}[row sep=large,column sep=large]
\{a\} \ar[r] \ar[d]& \{b,c\}\ar[d]\\
\{d\} \ar[r] & \{e,f\}
\end{tikzcd}
\ese
The vertical arrows can be taken to be equivalences (bijections): there is a unique map $\{a\}\to\{d\}$ and we can define $\{b,c\}\to\{e,f\}$ by $b\mapsto e$, $c\mapsto f$. Then, choosing the horizontal maps to be $a\mapsto b$ and $d\mapsto f$ shows that the square need not be commutative.

Hence, let us assume that the square is commutative. Since $f$ is an equivalence, we have
\bse
j=j\circ \id_C=j\circ f\circ f^{-1}=g\circ i\circ f^{-1}.
\ese
Let $h\cl B\to Z$ and $k\cl C\to Z$ be morphisms such that $h\circ i=k\circ f$. Consider the morphism $h\circ g^{-1}\cl D\to Z$.
\bse
\begin{tikzcd}%[row sep=large,column sep=large]
A \ar[rr,"i"] \ar[dd,"f"]&& B\ar[dd,"g"]\ar[dddr,bend left=30,"h"]&\\
&&&\\
C \ar[rrrd,bend right=30,"k"']\ar[rr,"j"] && D\ar[dr,"h\circ g^{-1}"']&\\
&&&Z
\end{tikzcd}
\ese
We clearly have $(h\circ g^{-1})\circ g=h$ and
\bse
(h\circ g^{-1})\circ j=h\circ g^{-1}\circ (g\circ i\circ f^{-1})= h\circ i\circ f^{-1} =k \circ f \circ f^{-1}=k.
\ese
So $h\circ g^{-1}\cl D\to Z$ makes the two triangles in the diagram commute.

To see that it is the unique such, let $l\cl D\to Z$ be such that $l\circ j=k$ and $l\circ g=h$. Then
\bse
l=l\circ \id_D=l\circ (g\circ g^{-1})=(l\circ g)\circ g^{-1}=h\circ g^{-1}.
\ese
\item The dual of (a) is stated and proved in Problem 2.21. The dual of (b) reads: A commutative square
\bse
\begin{tikzcd}[row sep=large,column sep=large]
A \ar[r,"i"] \ar[d,"f"]& B\ar[d,"g"]\\
C \ar[r,"j"] & D
\end{tikzcd}
\ese
in which both $f$ and $g$ are equivalences, is a pullback square.

Indeed, let $h\cl X\to B$ and $k\cl X\to C$ be morphisms such that $g\circ h=j\circ k$. Consider the morphism $f^{-1}\circ k\cl X\to A$.
\bse
\begin{tikzcd}%[row sep=large,column sep=large]
X \ar[dr,"f^{-1}\circ k"] \ar[rrrd,bend left=30,"h"]\ar[dddr,bend right=30,"k"']&&&\\
&A \ar[rr,"i"] \ar[dd,"f"]&& B\ar[dd,"g"]\\
&&&\\
&C \ar[rr,"j"] && D
\end{tikzcd}
\ese
We clearly have $f\circ (f^{-1}\circ k)$ and, noting that $i=g^{-1}\circ j\circ f$,
\bse
i\circ(f^{-1}\circ k)=(g^{-1}\circ j\circ f)\circ(f^{-1}\circ k)= g^{-1}\circ j\circ k =g^{-1}\circ g\circ h=h.
\ese
For uniqueness, let $l\cl X\to A$ be such that $f\circ l=k$ and $i\circ l=h$. Then
\bse
l=\id_A\circ l=(f^{-1}\circ f) \circ l=f^{-1}\circ (f \circ l)=f^{-1}\circ k.
\ese
\een
\es

\bp
Prove Theorem 2.40: Consider the diagram
\bse
\begin{tikzcd}[row sep=small]
A_1 \ar[rr,"f_1"] \ar[dd,"h_1"]&& A_2 \ar[dd,"h_2"]\ar[rr,"f_2"]&& A_3\ar[dd,"h_3"]\\
 &(I)& &(II)&\\
B_1 \ar[rr,"g_1"]&& B_2\ar[rr,"g_2"] && B_3
\end{tikzcd}
\ese
and denote the outside square by $(T)$.
\ben[label=(\alph*)]
\item If $(I)$ and $(II)$ are pushouts, then $(T)$ is also a pushout.
\item If $(I)$ and $(T)$ are pushouts, then $(II)$ is also a pushout.
\een
\ep

\bs
\ben[label=(\alph*)]
\item Since $(I)$ and $(II)$ are pushouts, they are commutative squares. We have
\bse
h_3\circ (f_2\circ f_1) =(g_2\circ h_2)\circ f_1=g_2\circ (g_1\circ h_1)
\ese
and hence $(T)$ is also a commutative square. Let $f\cl A_3\to C$ and $g\cl B_1\to C$ be morphisms such that $f\circ(f_2\circ f_1)=g\circ h_1$.
\bse
\begin{tikzcd}[row sep=small]
A_1 \ar[rr,"f_1"] \ar[dd,"h_1"]&& A_2 \ar[dd,"h_2"]\ar[rr,"f_2"]&& A_3\ar[dd,"h_3"] \ar[rddd,bend left=30,"f"]&\\
 &(I)& &(II)&&\\
B_1 \ar[rrrrrd,bend right=25,"g"]\ar[rr,"g_1"]&& B_2\ar[drrr,bend right=10,dashed,"h"]\ar[rr,"g_2"] && B_3\ar[dr,dashed,"k"]&\\
&&&&& C
\end{tikzcd}
\ese
Since $(f\circ f_2)\circ f_1=g\circ h_1$, by the pushout property of $(I)$, there exists a unique $h\cl B_2\to C$ such that $h\circ g_1=g$ and $h\circ h_2=f\circ f_2$. The latter implies that there exists a unique $k\cl B_3\to C$ such that $k\circ g_2 = h$ and $k\circ h_3=f$ by the pushout property of $(II)$. Then $k\circ(g_2\circ g_1)=h\circ g_1=g$.

For uniqueness, let $k'\cl B_3\to C$ is such that $k'\circ h_3=f$ and $k'\circ (g_2\circ g_1)=g$. Consider the composite $k'\circ g_2\cl B_2\to C$. We have $(k'\circ g_2)\circ g_1=g$ and 
\bse
(k'\circ g_2)\circ h_2=k'\circ (h_3\circ f_2)=f\circ f_2,
\ese
but $h\cl B_2\to C$ is the unique morphism satisfying these equations and hence $k'\circ g_2 =h$. Since we also have $k'\circ h_3=f$ and $k$ is the unique morphism satisfying these equations, we must have $k=k'$. 
\item {\scshape Note}. The statement in (b) is not true without assuming that $(I)$ and $(II)$ are both commutative. Indeed, the commutativity of $(I)$ and $(T)$ does not imply that of $(II)$, as is shown, for instance, by the following diagram in $\mathbf{Sets}$. 
\bse
\begin{tikzcd}[row sep=large,column sep=large]
\{a\}\ar[r,"a\,\mapsto\,b"]\ar[d] & \{b,c\} \ar[d] \ar[r,"b\,\mapsto\, d","c\,\mapsto\, e"'] & \{d,e\} \ar[d,"e\,\mapsto\,i","d\,\mapsto\, h"']\\
\{f\} \ar[r] & \{g\} \ar[r,"g\,\mapsto \, h"] & \{h,i\}
\end{tikzcd}
\ese

Hence, assuming that $(I)$ and $(II)$ are both commutative, let $f\cl A_3\to C$ and $g\cl B_2\to C$ be morphisms such that $f\circ f_2=g\circ h_2$.
\bse
\begin{tikzcd}[row sep=small]
A_1 \ar[rr,"f_1"] \ar[dd,"h_1"]&& A_2 \ar[dd,"h_2"]\ar[rr,"f_2"]&& A_3\ar[dd,"h_3"] \ar[rddd,bend left=30,"f"]&\\
 &(I)& &(II)&&\\
B_1 \ar[rrrrrd,bend right=25,"g\circ g_1"']\ar[rr,"g_1"]&& B_2\ar[drrr,bend right=22,"g"'] \ar[drrr,dashed,bend right=5,"u"] \ar[rr,"g_2"] && B_3\ar[dr,dashed,"k"]&\\
&&&&& C
\end{tikzcd}
\ese
Since 
\bse
f\circ (f_2\circ f_1)=(g\circ h_2)\circ f_1=(g\circ g_1)\circ h_1,
\ese
by the pushout property of $(T)$, there is a unique morphism $k\cl B_3\to C$ such that $k\circ h_3=f$ and $k\circ(g_2\circ g_1)=g \circ g_1$.

Now consider the composites $g\circ h_2$ and $g\circ g_1$. By the commutativity of $(I)$, we have $(g\circ h_2)\circ f_1=(g\circ g_1)\circ h_1$. Hence, by the pushout property of $(I)$, there is a unique $u\cl B_2\to C$ such that
\bse
u\circ h_2=g\circ h_2 \qquad \text{ and }\qquad u\circ g_1=g\circ g_1.
\ese
Obviously, $g$ satisfies these equations, but we also have
\bse
(k\circ g_2)\circ g_1=k\circ(g_2\circ g_1)=g\circ g_1
\ese
and
\bse
(k\circ g_2)\circ h_2=k\circ(h_3\circ f_2)=f\circ f_2=g\circ h_2.
\ese
Therefore $k\circ g_2=u=g$. 

For uniqueness, let $k'\cl B_3\to C$ be such that $k'\circ h_3=f$ and $k'\circ g_2=g$. Then
\bse
k'\circ(g_2\circ g_1)=(k'\circ g_2)\circ g_1=g\circ g_1
\ese
and $k'\circ h_3=f$, but, by the pushout property of $(T)$, $k$ is the unique morphism satisfying these equations. Thus $k'=k$.
\een
\es




































\part{Semi-Formal Homotopy Theory}
\chapter{Categories of spaces}

\section{Spheres and Disks}



\section{Topological Spaces}

\bp
Prove that the map
\bse
\alpha \cl \mor_{\mathbf{Top}}(X \times Y,Z) \longrightarrow \mor_{\mathbf{Top}}(X, \mor_{\mathbf{Top}}(Y,Z))
\ese
given by the formula $\alpha(f) \cl x\mapsto \boxed{y\mapsto f(x, y)}$ is a bijection.
\ep

\bs
\es

\addtocounter{exercise}{1}
\bp
\ben[label=(\alph*)]
\item Explicitly describe the CW structure of $X \times I$ in terms of the CW structure of $X$.
\item Prove Lemma 3.30: Let $X$ be a CW complex, and give $I = [0, 1]$ the standard CW decomposition with two zero cells and one $1$-cell. Show that in the CW product decomposition,
\bse
(X \times I)_{n+1} = (X \times 0) \cup (X_n \times I) \cup (X \times 1) \subseteq X \times I.
\ese
\een
\ep

\bs
\ben[label=(\alph*)]
\item 
\item 
\een
\es


\bp
Suppose $A,X, Y \in \mathcal{T}_{\circ}$ and $A \subseteq X$. Show that $X/A \in \mathcal{T}_{\circ}$ and $X \times Y \in \mathcal{T}_{\circ}$.
\ep

\bs
\es

\bp
Maps $f \cl X \to Y$ and $h \cl Y \to Z$ in $\mathcal{T}_{\circ}$ induce functions $f^*: \map_{\circ}(Y,Z) \longrightarrow \map_{\circ}(X,Z)$ and $h_* \cl \map_{\circ}(X, Y ) \longrightarrow \map_{\circ}(X,Z)$ given by $f^*(g) = g\circ f$ and $h_*(g) = h\circ g$. Show that $f^*$ and $h_*$ are continuous.

{\scshape Hint}. Express the function you are interested in as a composition of the map $\circ$ with another function.
\ep

\bs
\es


































\chapter{Homotopy}

\section{Homotopy of Maps}

\bx
Define homotopy for maps in the category $\mathcal{T}_{(2)}$ of pairs.
\ex

\bs
Intuitively, a homotopy between maps $f,g\in\map_{(2)}((X,A),(Y,B))$ should be a path $H\cl I\to\map_{(2)}((X,A),(Y,B))$ from $f$ to $g$. However, we want $H$ to be a morphism in $\mathcal{T}_{(2)}$, but neither $I$ nor $\map_{(2)}((X,A),(Y,B))$ are objects in $\mathcal{T}_{(2)}$. We can remedy this by defining $H\cl f\simeq g$ to be a map
\bse
H\cl (I,\vn)\longrightarrow (\map_{(2)}((X,A),(Y,B)),\map_{\circ}(X,B))
\ese
such that $H(0)=f$ and $H(1)=g$.
\es

\bp
Let $X$ be a topological space, and let $x, y \in X$. Say that $x \sim y$ if $x$ and $y$ are in the same path component of $X$ (i.e., if there is a path $\alpha \cl I \to X$ with $\alpha(0) = x$ and $\alpha(1) = y$).
\ben[label=(\alph*)]
\item Show that $\sim$ is an equivalence relation.
\item Conclude that homotopy is an equivalence relation in $\mathcal{T}_{\circ}$ and in $\mathcal{T}_*$.
\een
\ep

\bs
\ben[label=(\alph*)]
\item 
\item 
\een
\es

\bx
The set $[X, Y ]$ is a pointed set---what is the basepoint?
\ex

\bs
\es

\bp
Let $f \cl X \to Y$ and $g \cl Y \to Z$ be two maps in $\mathcal{T}_*$.
\ben[label=(\alph*)]
\item Show that if $x \sim x' \in X$, then $f(x) \sim f(x')$ in $Y$.
\item Define $\pi_0(f)$ to make $\pi_0$ a covariant functor from $\mathcal{T}_*$ to the category $\mathbf{Sets}_*$ of pointed sets.
\item Show that if $f \simeq g$ in $\mathcal{T}$, then $\pi_0(f) = \pi_0(g)$.
\een
\ep

\bs
\ben[label=(\alph*)]
\item 
\item 
\item 
\een
\es

\bx
Show that if $F$ respects homotopy and $G$ is a homotopy functor, then $G \circ F$ is a homotopy functor.
\ex

\bs
\es

\bp
\ben[label=(\alph*)]
\item Show that the functors $- \times Z$, $ - \rtimes -$, $-\wedge Z$ and $\map_*(- ,- )$ respect homotopy in $\mathcal{T}_*$. 
\item Show that the functors $-\times Z$, $Z\rtimes -$ and $\map_{\circ}(-,-)$ respect homotopy in $\mathcal{T}_{\circ}$. 
\een
\ep

\bs
\ben[label=(\alph*)]
\item 
\item 
\een
\es

\bp
\ben[label=(\alph*)]
\item Interpret the homotopy set $[ - , - ]$ in terms of $\pi_0$.
\item  Is $[ - , - ]$ functorial? Is it a homotopy functor? Keep in mind that there are two variables.
\een 
\ep

\bs
\ben[label=(\alph*)]
\item 
\item 
\een
\es

\bx
Interpret $\pi_0$ in terms of $[ - , - ]$.
\ex

\bs
\es

\bp
Show that the definitions of free and pointed homotopy in terms of cylinders are equivalent to those given in Section 4.1.1.
\ep

\bs
\es

\bp
Using the definition of this section, show that
\ben[label=(\alph*)]
\item homotopy is an equivalence relation and 
\item if $f \simeq \bar{f}$ and $g \simeq \bar{g}$, then $g \circ f\simeq \bar{g}\circ \bar{f}$ in both the pointed and unpointed contexts.
\een
\ep

\bs
\ben[label=(\alph*)]
\item 
\item 
\een
\es

\bx
Show that if $X$ is path-connected, then any two paths are freely homotopic. If the paths are pointed, then they are pointed homotopic.
\ex

\bs
\es

\bp
\ben[label=(\alph*)]
\item Suppose there are path homotopies $\alpha_1 \simeq \alpha_2$ and $\beta_1 \simeq \beta_2$. Show that $\alpha_1 * \beta_1 \simeq \alpha_2 * \beta_2$, assuming the concatenation is defined.
\item Show that $\alpha * \overleftarrow{\alpha}\simeq *$ is path homotopic to the constant path at $\alpha(0)$.

{\scshape Hint}. For (b), consider the path which follows $\alpha$ from time $0$ to time $2t$, then sits still until time $2-2t$, after which it follows $\overleftarrow{\alpha}$ until time $1$.
\een
\ep

\bs
\ben[label=(\alph*)]
\item 
\item 
\een
\es

\bp
Let $p \cl I \to I$ be any path from $0$ to $1$.
\ben[label=(\alph*)]
\item Show that $p$ is path homotopic to $\id_I$.
\item Show that $\alpha \circ p$ is path homotopic to $\alpha$. 
\een
{\scshape Hint}. Look ahead to Section 4.2.1.
\ep

\bs
\ben[label=(\alph*)]
\item 
\item 
\een
\es

\bp
\ben[label=(\alph*)]
\item Show that the paths $(\alpha * \beta) * \gamma$ and $\alpha * (\beta * \gamma)$ are path homotopic.
\item Let $\alpha$ be a path in $X$ from $x$ to $y$, and let $\boxed{x}$ and $\boxed{y}$ be the constant paths at $x$ and at $y$, respectively. Show that there are path homotopies $\alpha * \boxed{y} \simeq \alpha \simeq \boxed{x} * \alpha$.
\een
{\scshape Hint}. The paths in question are reparametrisations of each other.
\ep

\bs
\ben[label=(\alph*)]
\item 
\item 
\een
\es

\bx
Show that the paths $(\alpha * \beta) * \gamma$ and $\alpha * (\beta * \gamma)$ are equal if and only if all three paths are constant.
\ex

\bs
\es

\bp
Define reparametrisation of homotopies, and show that there is always a homotopy-of-homotopies from $H$ to any reparametrisation of $H$.
\ep

\bs
\es

\bp
Let $H \cl f \simeq g$ be a homotopy.
\ben[label=(\alph*)]
\item Show that $H * \boxed{g} \simeq H \simeq \boxed{f} * H$.
\item Show that there is a homotopy-of-homotopies $H * \overleftarrow{H}\simeq\boxed{f}$.
\een
\ep

\bs
\ben[label=(\alph*)]
\item 
\item 
\een
\es


















\part{Four Topological Inputs}
\chapter{Subdivision of Disks}

\section{The Seifert-Van Kampen Theorem}

\addtocounter{exercise}{1}
\bx
Show that the free product of two groups is very poorly named: it is the sum in the category of groups.
\ex

\bs
Consider the free product $A*B$, defined as the free group on $A\sqcup B$ subject to the relations already present in $A$ and $B$, and with $e_A$ and $e_B$ identified. Its elements are words of the form $a_1^{k_1}b_1^{k_2}\cdots a_n^{k_{n-1}}b_n^{k_{2n}}$ with $a_1,\ldots,a_n\in A$, $b_1,\ldots,b_n\in B$, and $k_1,\ldots,k_{2n}\in \Z$. Define $i_A\cl A \to A*B$ and $i_B\cl B\to A*B$ by sending $a\in A$ and $b\in B$ to the one-letter words $a$ and $b$, respectively.

Let $D$ be a group and let $j_A\cl A\to D$ and $j_B\cl B\to D$ be group homomorphisms. Define a map $\phi\cl A*B\to D$ by
\bse
\phi(a_1^{k_1}b_1^{k_2}\cdots a_n^{k_{2n-1}}b_n^{k_{2n}}):=j_A(a_1^{k_1})j_B(b_1^{k_2})\cdots j_A(a_n^{k_{2n-1}})j_B(b_n^{k_{2n}}).
\ese
Since $j_A$ and $j_B$ are group homomorphisms, so is $\phi$. Furthermore, for each $a\in A$, we have
\bse
\phi\circ i_A(a)=j_A(a),
\ese
so $\phi\circ i_A=j_A$ and, similarly, $\phi\circ i_B=j_B$. 
Finally, suppose that $\psi\cl A*B\to D$ is a group homomorphism such that $\psi\circ i_A=j_A$ and $\psi\circ i_B=j_B$. Then
\bi{rCl}
\psi(a_1^{k_1}b_1^{k_2}\cdots a_n^{k_{2n-1}}b_n^{k_{2n}}) & = &\psi(a_1)^{k_1}\psi(b_1)^{k_2}\cdots \psi(a_n)^{k_{2n-1}}\psi(b_n)^{k_{2n}}\\
& = &\psi(i_A(a_1))^{k_1}\psi(i_B(b_1))^{k_2}\cdots \psi(i_A(a_n))^{k_{2n-1}}\psi(i_B(b_n))^{k_{2n}}\\
& = & j_A(a_1)^{k_1}j_B(b_1)^{k_2}\cdots j_A(a_n)^{k_{2n-1}}j_B(b_n)^{k_{2n}}\\
&=&\phi(a_1^{k_1}b_1^{k_2}\cdots a_n^{k_{2n-1}}b_n^{k_{2n}}).
\ei
So $\psi=\phi$ and hence $(A*B,i_A,i_B)$ is a co-product of $A$ and $B$.
\es





\appendix
\chapter{Some Algebra}
\section{Modules, Algebras and Tensor Products}







\section{Exact Sequences}

























\end{document}





























